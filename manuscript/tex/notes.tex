\chapter{Notes}


\notenuminnotes{1} \emph{A Dhammapada For Contemplation}, Aruno Publications, 2017.

https://forestsangha.org/teachings/books/a-dhammapada-for-contemplation?language=English

\notenuminnotes{2} \emph{Ajahn Tate} (1902-1994)

One of the most influential Thai Forest Tradition monks of the last century.

http://www.accesstoinsight.org/lib/thai/thate/thateauto.html

\notenuminnotes{3} \emph{Mahasaropama Sutta}

http://www.accesstoinsight.org/tipitaka/mn/mn.029.than.html

\notenuminnotes{4} \emph{Mahamangala Sutta}

http://www.accesstoinsight.org/tipitaka/kn/snp/snp.2.04.nara.html

\notenuminnotes{5} \emph{Dhutanga vatta}

TODO

TAN GAMBHIRO CAN YOU KINDLY FILL IN THE 13

\notenuminnotes{6} \emph{Five spiritual faculties}

Saddha, Viriya, Sati, Samadhi, Pañña (Faith, Energy/Interest, Mindfulness, Concentration, Discernment)

TODO

TAN GAMBHIRO CAN YOU KINDLY FILL IN diacriticals

\notenuminnotes{7} \emph{Master Hsu Yun} (1840-1959)

One of the most influential Chinese Buddhists of the last two centuries. See his
autobiography Empty Cloud. Died aged 119 years.

\notenuminnotes{8} \emph{Seeing the Way, Volume 2}

https://forestsangha.org/teachings/books/seeing-the-way-vol-two?language=English

\notenuminnotes{9} \emph{Four Noble Truths}

https://forestsangha.org/teachings/books/the-four-noble-truths?language=English

\notenuminnotes{10} \emph{Attahi athano natho\ldots{}}

TODO

\notenuminnotes{11} \emph{Inner Listening}

For further details on using this meditation object, see \emph{Inner Listening} by Ajahn Amaro:

https://forestsangha.org/teachings/books/inner-listening?language=English

\notenuminnotes{12} \emph{Dhammapada verse 276}

TODO verse formatting

The Awakened Ones can but point the way; we must make the effort ourselves.
Those who reflect wisely and enter the path are freed from the fetters of Mara.

\notenuminnotes{13} \emph{The Gospel of Consumption}

https://orionmagazine.org/article/the-gospel-of-consumption/

\notenuminnotes{14} \emph{The Four Right Efforts}

\begin{itemize}
\item The effort to establish as yet unarisen wholesome states of mind.
\item The effort to protect already arisen wholesome states of mind.
\item The effort to remove already arisen unwholesome states of mind.
\item The effort to avoid the arising of as yet unarisen unwholesome states of mind.
\end{itemize}

\notenuminnotes{15} \emph{Sound of Silence} by Ajahn Sumedho

https://cdn.amaravati.org/wp-content/uploads/2014/10/Ajahn-Sumedho-Volume-4-The-Sound-of-Silence.pdf

\notenuminnotes{16} \emph{Eight Worldly Dhammas: Lokavipatti Sutta}

https://www.accesstoinsight.org/tipitaka/an/an08/an08.006.than.html

\notenuminnotes{17} \emph{Mindfulness of Breathing Sutta}

https://www.accesstoinsight.org/tipitaka/mn/mn.118.than.html

