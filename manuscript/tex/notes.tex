\begin{thebibliography}{9}

\bibitem{dhammapada-aruno} \emph{A Dhammapada For Contemplation}, Aruno Publications, 2017

  {\urlsize \url{https://forestsangha.org/teachings/books/a-dhammapada-for-contemplation?language=English}}

\bibitem{ajahn-tate} \emph{Ajahn Thate} (1902-1994)

  One of the most influential Thai Forest Tradition monks of the last century.

  {\urlsize \url{http://www.accesstoinsight.org/lib/thai/thate/thateauto.html}}

\bibitem{mahasaropama-sutta} \emph{MN 29, Mahāsāropama Sutta: The Longer Heartwood-simile Discourse}

  {\urlsize \url{http://www.accesstoinsight.org/tipitaka/mn/mn.029.than.html}}

\bibitem{mahamangala-sutta} \emph{Snp 2.4, Mahā-Maṅgala Sutta: Blessings}

  {\urlsize \url{http://www.accesstoinsight.org/tipitaka/kn/snp/snp.2.04.nara.html}}

\bibitem{dhutanga} \emph{Dhutaṅga vaṭṭa}

  Voluntary ascetic practices that practitioners may undertake from time to time
  or as a long-term commitment in order to cultivate renunciation and
  contentment, and to stir up energy. For the monks, there are thirteen such
  practices: (1) using only patched-up robes; (2) using only one set of three
  robes; (3) going for alms; (4) not by-passing any donors on one’s alms path;
  (5) eating no more than one meal a day; (6) eating only from the alms-bowl;
  (7) refusing any food offered after the almsround; (8) living in the forest;
  (9) living under a tree; (10) living under the open sky; (11) living in a
  cemetery; (12) being content with whatever dwelling one has; (13) not lying
  down.

\bibitem{faculties} \emph{Five spiritual faculties}

  \emph{Saddhā} (faith, conviction), \emph{viriya} (persistence, energy,
  interest), \emph{sati} (mindfulness), \emph{samādhi} (concentration), and
  \emph{paññā} (discernment).

\bibitem{hsu-yun} \emph{Master Hsu Yun} (1840-1959)

  One of the most influential Chinese Buddhists of the last two centuries. See
  his autobiography Empty Cloud. Died aged 119 years.

\bibitem{seeing-vol2} \emph{Seeing the Way, Volume 2}

  {\urlsize \url{https://forestsangha.org/teachings/books/seeing-the-way-vol-two?language=English}}

\bibitem{four-noble-truths} \emph{The Four Noble Truths} by Ajahn Sumedho

  {\urlsize \url{https://forestsangha.org/teachings/books/the-four-noble-truths?language=English}}

\bibitem{dhp-attahi} \emph{Dhammapada verse 160, ``Attā hi attano nātho\ldots{}''}

  Truly it is ourselves\\
  that we depend upon;\\
  how could we really\\
  depend upon another?\\
  When we reach the state\\
  of self-reliance\\
  we find a rare refuge.

\bibitem{inner-listening} \emph{Inner Listening} by Ajahn Amaro

  {\urlsize \url{https://forestsangha.org/teachings/books/inner-listening?language=English}}

\clearpage

\bibitem{dhp-276} \emph{Dhammapada verse 276}

  The Awakened Ones\\
  can but point the way;\\
  we must make the effort ourselves.\\
  Those who reflect wisely\\
  and enter the path are freed\\
  from the fetters of Mara.

\bibitem{gospel} \emph{The Gospel of Consumption}

  {\urlsize \url{https://orionmagazine.org/article/the-gospel-of-consumption/}}

\bibitem{right-effort} \emph{The Four Right Efforts}

  \begin{itemize}
  \item The effort to establish as yet unarisen\\ wholesome states of mind.
  \item The effort to protect already arisen\\ wholesome states of mind.
  \item The effort to remove already arisen\\ unwholesome states of mind.
  \item The effort to avoid the arising of as yet unarisen\\ unwholesome states of mind.
  \end{itemize}

\bibitem{aj-sumedho-sound-of-silence} \emph{The Sound of Silence} by Ajahn Sumedho

  {\urlsize \url{https://forestsangha.org/teachings/books/anthology-vol-4-the-sound-of-silence?language=English}}

\bibitem{wordly-dhammas} \emph{AN 8.6, Lokavipatti Sutta: The Failings of the World}

  The Eight Worldly Dhammas are described as: gain and loss, status and
  disgrace, praise and blame, pleasure and pain.

  \mbox{\urlsize \url{https://www.accesstoinsight.org/tipitaka/an/an08/an08.006.than.html}}

\bibitem{anapanasati} \emph{MN 118, Ānāpānasati Sutta: Mindfulness of Breathing}

  {\urlsize \url{https://www.accesstoinsight.org/tipitaka/mn/mn.118.than.html}}

\end{thebibliography}