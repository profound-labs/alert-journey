\chapterTitleVertical{U\\N\\A\\F\\R\\A\\I\\D\par T\\O\par C\\A\\R\\E}

\chapter{Unafraid To Care}

\tocChapterNote{Relationships, loss, loneliness, open-heartedness, service,
concepts, frustration, spiritual toolkit, sense restraint, wise reflection,
sound of silence, contemplative enquiry.}

The feelings we have when we care for another occupy a special place in
our hearts. To care deeply for another feels like a blessing. Remember
how the Buddha taught about cultivating loving-kindness: he asks us to
hold in our minds an image of a mother with her only child. He wants us
to focus on the mother’s regard for her child and give rise to a
heartfelt appreciation for the beauty of selfless caring.

Similarly, the feeling of being cared for is a truly wonderful thing. To
receive caring from another is like receiving warmth when we are cold,
or receiving food when we are hungry. It is a type of nourishment. And
it is an essential nourishment.

Now let’s look at what may obscure the beauty of caring. We are all
familiar with the way in which affection leads to hurt when a
relationship is damaged or lost. If we’ve experienced such suffering
after having met the Buddha’s teachings on the Four Noble
Truths\cite{four-noble-truths},
we have probably wondered whether it is really possible both to care deeply
\emph{and} be free from suffering. Can we care without becoming caught in
clinging and leaving ourselves vulnerable to agonizing heartache?

\section{Avoidance Strategies}

After being hurt in a relationship, some decide never to leave
themselves open to such suffering again. They choose a strategy of
closing their hearts as a defence, making themselves unavailable for
trusting relationships of any kind. It is understandable that we try to
protect ourselves from suffering, but in this case the strategy leads to
another kind of suffering – that of isolation, loneliness.

Then there are those who work in what we call the caring professions,
who fall prey to a kind of emotional burnout because they find they are
required to carry more pain than their hearts could bear. Although they
might well have started out in the profession as genuinely compassionate
and caring, really wanting to make a difference, over time they find
they have become cold-hearted and cynical. That is very sad.

Others embark on a spiritual journey after experiencing more hurt than
they could handle, hoping to transcend it all. But if they don’t
understand ‘transcendence’ in the way the Buddha meant it, after years
of discipline and meditation they could find themselves in a narrow
cul-de-sac, facing an intimidating wall of denied suffering. If they are
fortunate, they’ll eventually realize that the Buddha meant what he said
about mindfulness of suffering leading to freedom from suffering. Our
attempts to avoid suffering are definitely not the way. It is no
exaggeration to say that the spiritual journey in fact begins when we
find ourselves face to face with that which we were trying to escape. Up
to that point, we are still only preparing for the journey.

The Buddha’s teachings on cultivating caring aim to equip us with
skilful defences, which strengthen and protect us without compromising
our sensitivity and discernment. They aim to give us the skill to meet
life, whatever that might mean in our case. Nobody goes through this
experience of life without periods of feeling intensely challenged. What
matters is how well prepared we are for the challenges. If our heart
faculties are not adequately developed, if our spiritual education isn’t
sufficient, it is understandable that we feel ill-equipped. However,
feeling ill-equipped doesn’t exempt us from having to face the
consequences of storing away our suffering in unawareness. Accumulated
unlived life doesn’t simply disappear; it lurks in the basement,
festering. The fumes emitted by our stash of unacknowledged suffering
can have the effect of desensitizing us, leading to dispiriting
cynicism. If we have even an inkling that this could be true, it makes
sense to stop avoiding and begin looking for ways to start clearing out
the basement. Don’t be deterred by concerns about how long it will take.
The Buddha encourages us to just get started. Many unexpected means of
support will be offered to us along the way.

\section{Attending To The Well-being Of Others}

The Buddha clearly supported the cultivation of meditation in solitude,
but not at the expense of sensitivity to the well-being of others. There
was an occasion when he visited a group of forest-dwelling monks and
expressed his admiration for their evident degree of cooperation. On
this occasion he asked one of the monks, Venerable Anuruddha, how they
were getting on living together, and Anuruddha replied that because
their actions of body, speech and mind were intentionally imbued with
loving-kindness, they lived together in a mutually beneficial,
harmonious atmosphere.

The abbot of one of our other branch monasteries recently told me how a
young monk there had spontaneously approached him to express gratitude
for the time he had spent performing what we call \emph{upatak} duties. Being
an \emph{upatak} is akin to being a personal attendant to a senior Sangha
member. Ajahn Chah always insisted that in his monasteries this two and
a half thousand year-old tradition should be maintained. Everybody,
including the Westerners, was expected to observe it. ‘Attending to the
teacher’ was considered the best way for those senior in the training to
share the benefits accumulated with those younger. Ajahn Chah praised a
lifelong commitment to meditation, but he also gave special emphasis to
developing and maintaining harmonious, cooperative community. The
technical Pali term is \emph{saddhivihāraka}, ‘one who lives along with’. One
young monk who arrived at Ajahn Chah’s monastery had previously been
acquainted only with monasteries where solitary meditation was
emphasized. When this monk was invited to take up \emph{upatak} duties, which
in effect meant waiting on the teacher, that didn’t immediately appeal
to him. ‘Why can’t the abbot look after himself? Why does he need
someone else to take care of him? He’s not old.’ It was only after some
time that he came to see for himself the point of consciously caring for
another.

When we hear that the Buddha taught, ‘You should be your own refuge; how could
another be your refuge?’\cite{dhp-attahi},
we might assume he is saying that
we should focus on paying attention to ourselves. We need to listen
really carefully to what the Buddha is saying here; if we don’t, we
might compound our suffering, making it even more difficult to deal
with. Being our own refuge does not mean closing ourselves off from
others and caring only about ‘me’. To be truly our own refuge means to
completely let go of ourselves through right understanding. Cultivating
a heart of caring, a heart of well-wishing for all beings, nurtures such
understanding. There is a rare beauty to be found in the ability to
surrender ‘me’ and ‘my way’, and to tune into the needs of others.

\section{Suffering Is A Choice}

The question of what has the power to obstruct the beauty of caring
pertains not just to our relationships with people, but also to the way
we relate to things, and to views and opinions. Perhaps for instance, we
thought that we were being compassionate towards planet earth, taking
good care of her, only to catch ourselves behaving aggressively towards
those we see as exploiting her. Can we tolerate having our views and
opinions contradicted without acting aggressively? This is not to say
that we should never have feelings of aversion when we witness abuse. To
say so would be like saying we shouldn’t have an immune system; our
immune system is not supposed to be passive. However, we do need to be
extremely careful that an appropriate sense of aversion doesn’t turn
into a thoroughly inappropriate reaction of hatred, causing harm to
ourselves and others. Feeling aversion can be functional, but if we
cling to that feeling it becomes something more; it becomes ill-will.
Once we are possessed by ill-will our faculty of discernment is
compromised and we can no longer trust ourselves to make balanced
decisions.

Experiencing loss on any level easily leads to hurt. But do those hurt
feelings \emph{have} to proliferate into being caught in a state of
negativity? It is worth looking closely to see if we unconsciously hold
to such a view. There is an adage in our culture which is sometimes
heard at times of coming to terms with the pain of loss: ‘Suffering is
the price we pay for having loved’. The implication is that suffering is
inescapable if we really care about anyone or anything. Surely to accept
such a view conditions us into a fear of wholehearted caring. That can’t
be the way!

One of the most important principles which the Buddha’s awakening
revealed to us is that we are not obliged to suffer; suffering is a
choice. It is true that all beings experience pain, both those who are
awakened and those who are not; but turning pain into suffering is
something extra that we \emph{do}. Physical pain or emotional pain, subtle
pain or gross pain, are all part of being human. When we resist pain out
of unawareness and cling, we are adding to it, we complicate it, we are
causing the suffering. The Buddha’s teachings are an invitation to
question, to enquire, and find out for ourselves whether it is true that
we can’t care without being caught in clinging.

\section{Contemplative Enquiry}

How then do we enquire into this conundrum of wanting to live with
caring and sensitivity, while at the same time honouring the Buddha’s
awakening, remembering that suffering is not an obligation? We could
read up on what psychologists have to say on the subject and look at the
experiments conducted on attachment theory, fear of rejection and so on.
Or we could discuss the subject with friends and companions. But here,
in the context of shared contemplative enquiry, how are we to engage the
issue effectively?

The fact that we can even begin to think about these matters shows we
already have skill in using concepts. It is significant that we can
formulate such ideas as ‘caring’, ‘clinging’, ‘attachment’,
‘non-attachment’, ‘detachment’, ‘dissociation’, etc. One aspect of
contemplative enquiry is building on these skills by investigating in
detail the relationship we have with concepts. We don’t just \emph{use}
concepts, we look carefully at \emph{how} we use them. It is essential to
understand that the concepts we have about clinging are not the same as
the actuality of clinging. Our concepts about caring are not the same
thing as actually caring. We must always keep this in mind: so long as
we are working on the level of concepts, we are dealing with
approximations, not the real thing.

A related example: I have heard eloquent talks on the subject of
emotional intelligence, but found it difficult to sense the being behind
the persona that delivered the talk. And I recall an occasion when I
received some enthusiastic but not particularly peaceful persuasion from
someone who thought I should sign up of a course on non-violent
communication. We should not assume that theory and practice are the
same reality. Thinking and what we are thinking about are not the same
thing. This sounds so simple that it should hardly bear mentioning– but
it does bear mentioning, because it matters very much.

Our everyday level of thinking gives access to the initial stages of
enquiry. This gets the process started, but it won’t take us very far.
To equate thinking about clinging with the reality of clinging is like
mistaking a printed picture of a mango for an actual mango. Obviously we
would not make that mistake regarding a printed picture. No matter how
good the camera or subtle the use of Photoshop in adding highlights and
tweaking contrast, or the quality of the printer, we would never be
tempted to eat the printed picture. But we do consume concepts,
mistaking them for more than they actually are.

It has many times been reported that Ajahn Chah discouraged his
disciples from reading too much. ‘Just read the Books of Discipline;
that’s enough for now. Once you have practised, the true meaning of the
recorded teachings by the Buddha, the \emph{suttas}, will be clear.’ His view
was that too much reading resulted in accumulating too many concepts,
just more knowledge \emph{about} things, which wouldn’t necessarily help.
‘The reason you don’t actually know anything is because you know so
much. If you read the \emph{suttas} after having learnt to read your own
heart it will be like eating the dessert after the main course.’

Our concepts \emph{about} caring and clinging must be understood as
abstractions on the realities of caring and clinging. If we couldn’t use
concepts we would be in trouble, but when we do use them we must
remember their inherent limitations. If we forget and start assuming
that ideas are something more than symbolic representations, we should
expect an increase in frustration as our efforts fail to resolve our
suffering. We need to do more than merely think about these subjects.

\section{Our Spiritual Toolkit}

The result we are looking for in contemplative enquiry is the
understanding that actually resolves suffering. To arrive at such
understanding requires skill in using the tools in our spiritual
toolkit. It might also mean we need to acquire more tools. As with any
task, if we don’t have the right equipment, we can’t do the work. If we
don’t have access to modes of investigation any more subtle than
common-or-garden thinking, we will be disappointed in our efforts. This
is what the spiritual exercises of meditation and wise reflection are
for: they introduce us to more subtle ways of working with the dynamics
of our inner worlds.

There are many ways of talking about the tools required to apply
ourselves competently to the inner work. Different teachers will share
according to what they have found has worked for them. In my experience
there are three main tools: mindfulness (\emph{sati}), sense restraint
(\emph{indriya saṃvara}) and wise reflection (\emph{yoniso manasikara}). We could
also speak in terms of the five Spiritual Faculties: confidence
(\emph{saddhā}), vitality (\emph{viriya}), mindfulness (\emph{sati}), collectedness
(\emph{samādhi}) and discernment (\emph{paññā}), but in this teaching I would like
to stay with the first set of tools. Mindfulness is to do with the
quality of watchfulness. An image the Buddha gave to help us appreciate
mindfulness was that of a gatekeeper, standing alert at the gate to the
city, observing the comings and goings. Or we could think of the doorman
at a hotel, watching who comes in and who goes out. The doorman doesn’t
carry the bags up to the rooms or leave with a guest in a taxi. He stays
watchfully at the entrance to the hotel.

Sense restraint is the ability to set boundaries and keep to them. I
mentioned earlier our body’s immune system, which has the function of
saying no to agents of disease that threaten to disrupt our physical
health. We also need to be able to say no to any excessive exuberance
that threatens to disrupt our hearts. Excessive exuberance shows itself
in our tendency to become lost by either following the feelings which
arise when we meet sense objects – sights, sounds, scents, flavours,
sensations and ideas – or denying them. These are the two extreme
reactions. When sense restraint is well developed we have an ability to
contain reactions, neither following nor denying them. Thus the feelings
which arise with sense contact are available for investigation, and we
don’t have to be intimidated by sense objects, the attractive, the
repulsive or anything in-between.

Wise reflection is what we do with the new-found perspective on the
inner landscape. The benefit of exercising mindfulness and sense
restraint is an increase in inner awareness. We start to see in ways we
didn’t even suspect were possible before; we start to understand what
our teachers meant when they encouraged us to read our own hearts. When
the faculties of mindfulness and sense restraint are not adequately
developed, we have difficulty in seeing what it is that keeps tripping
us up. When they have been adequately developed, wise reflection can do
its work, which is to look more deeply, to listen more accurately,
beyond the surface appearance of things. Wise reflection loves looking
for and finding the most relevant questions to ask, those questions
which begin to ease the tension and actually resolve our suffering.

\section{Proficiency In Meditation}

Just as all beings long to be free from suffering, so our hearts long
to know truth. When we have developed some skill in using the tools in
our spiritual toolkit, we can feel more confident in our practice of
meditation. Whether we are developing mindfulness of breathing, focusing
on listening to the sound of silence
(sometimes also called
\emph{nada})\cite{inner-listening},
dwelling on the theme of loving-kindness or using any other of the
many modes of disciplining attention, our practice only prospers once we
have an embodied appreciation for how the spiritual tools are to be
used. However much we might have read about them and how they might be
applied, until we put them to use they are like money sitting untouched
in a bank account: they have potential but their value has not been
realized.

Much has been said by others about mindfulness of breathing, so here I
will just say something about working with what Luang Por Sumedho has
called the ‘sound of silence’. Obviously this is a poetic reference to
the meditation object in question; of course, true silence has no sound.
But as with silence, this sound is always present. In my imagination it
is the sound you would hear if you were to wander through a grove of
aspen trees which were made of silver; it is what it would sound like as
a gentle breeze made the thousands of small, silver, aspen leaves
flutter. Not everybody finds they can tune into this sound, but for
those who can, this high-pitched ringing is always there behind whatever
other sounds we might be hearing, whether the sound of inner thinking or
sounds from outside. It has gentleness, harmony, and beauty; it is
natural, not fabricated. Attending to this ‘sound of silence’ gives rise
to a very helpful frame of reference.

For some meditators the habitual ‘controlling’ which has become
associated with making effort seems to infect their attention. When they
‘pay attention’ to the body breathing, they can’t help but interfere
with it. For them there is no such thing as being mindful of natural
breathing in and breathing out. Everything is disturbed by compulsive
controlling, including the rhythm of the breath. To discover that the
meditation object of the sound of silence remains undisturbed,
regardless of our habitual tendencies, can be a great relief.

\section{Being Here On Time}

One of the very important insights meditation can give us, even early on
in practice, is that all the activity of our minds is not who and what
we are. We don’t have to have been meditating for many years to see
this. It is tragic that most people believe they are just the activity
of their minds, their thoughts and feelings, and hence the turmoil of
their lives. But once we get a sense of the space within which all this
activity is taking place, or the silence out of which all the inner
sounds are arising, we naturally start to relax. We begin to see that
none of this activity is ultimate. None of it! Not the agreeable nor the
disagreeable, not the acceptable nor the unacceptable. This insight
gives us an altogether different perspective, a whole new way of
relating to life. Now, when disagreeable sense objects impinge upon our
senses, we can study the process; we are not obliged to react.

One year during a period of monastic retreat here at Harnham, we made an
exception and allowed a visitor to join us. We are usually rather
protective of these periods of structured silence, but on this occasion
there was a good reason to make an exception. It turned out that this
guest was particularly noisy and I started having regrets. But I recall
one afternoon, as we were sitting together in the hall, a brief instant
that affirmed the point we are considering here. I was sitting facing
the shrine, as I usually do during these retreats; the hall was quite
silent. Anyone who had to move during the sitting would do so carefully
and quietly. But not this guest. What happened on that occasion however,
turned out to be a gift. In the midst of the silence there came a loud
‘clunk’ as the guest moved the meditation stool off to the side and onto
the wooden floorboards. As it happened, at that moment I had enough
preparedness, enough mindfulness, enough sense-restraint, to be able to
catch what was about to happen before it happened. The sound of the
stool hitting the floor was the sound of the stool, I couldn’t stop
that; but significantly, I noticed that I didn’t have to follow the
inclination to react with annoyance. I had a choice whether or not to
follow the inclination. It was a very brief moment, but with beneficial
consequences. In such a situation, if we remain abiding with quiet
watchfulness, the mind does not become disturbed. That doesn’t mean we
cease feeling what we feel; we feel what we feel, but with greater
accuracy. Because the mind is unperturbed, our discernment faculties are
unobscured and available to serve the situation according to what is
skilful.

This applies similarly when agreeable sense objects impinge upon our
senses; we can study the process, but we are not obliged to react. When
the pleasurable feelings which are associated with caring appear, if we
are prepared, if we have a good enough level of skill in using these
spiritual tools, we will sense the space around those feelings and see
that we have a choice: to abandon abiding \emph{as} awareness and follow the
feelings, or feel the feelings, fully, and allow wisdom and compassion
to determine any action.

We should bear in mind that what we are aiming for is a good enough
level of skill. We don’t have to be the champion cyclist who wins the
Tour de France; it is good enough to have the skill needed to enjoy a
gentle cycle-ride through the countryside. Being over-zealous in our
investigations obscures the subtlety that is required.

\section{The Right Amount}

This level of investigation is more refined, more subtle than what for
most of us would have be usual. Previously when we were faced with a
question like, ‘Is it possible to care without creating more
suffering?’, we would likely have reverted to a coarse level of
thinking, a kind of internal verbal dialogue. Now we are investigating
in a feeling way. Contemplative enquiry is feeling enquiry. This is not
saying we are merely looking into feelings, although it could include
that; rather, it means we are working with a facility for feeling into
and around the activity in our hearts and minds, an enquiry that takes
place without thinking. This probably isn’t a facility we would have
been taught at school. This way of functioning is not generally
available without disciplined attention. And here the discipline we are
talking about is grounded in the self-respect that comes with living a
life of integrity.

With this upgraded set of tools we are now equipped to meet the really
challenging questions life offers us, with confidence and unapologetic
interest. If nobody else is interested in the questions which our heart
is asking, it doesn’t matter. We all have our own questions, our own
personal conundrums, and it is these precious questions that have the
power to awaken us. Though we must keep reminding ourselves of it, we
are no longer interested in simply appeasing the pain that life’s
troubling questions generate. Now we are interested in learning the
skill of receiving these questions so that they show us the way to be
free. We are not conjuring up just another concept which counters or
replaces one fixed mental position with another. We are not arguing with
ourselves until we agree to believe in some mental construct. Mental
arguments are powerless when it comes to opening the doorways to the
inner dimensions in which we feel free to feel whatever we feel, without
being obstructed by those feelings.

Back to our original question: is it possible to care without clinging?
The Buddha and our teachers care completely, with all their hearts,
holding nothing back, because they know reality. Their understanding
meant they could afford to give themselves to caring fully; they had
seen beyond all doubt that clinging is not necessary and suffering is
not an obligation. So instead of asking whether it is possible to care
without clinging, we should be asking whether we can be \emph{here} quickly
enough to catch the clinging before it happens. It is the preparedness
that matters.

Thank you for your attention.
