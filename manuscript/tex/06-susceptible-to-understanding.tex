\chapterNote{%
  ``A mind that is fit to receive teachings is a mind that is at ease, sensitive
  and open to hearing something new.''}

\chapter{Susceptible to Understanding}

\chapterTitleVertical{S\\U\\S\\C\\E\\P\\T\\I\\B\\L\\E\par T\\O\par U\\N\\D\\E\\R\\S\\T\\A\\N\\D\\I\\N\\G}

\tocChapterNote{Listening deeply, Dhamma talks, relaxation, contemplation,
discriminative faculties, inner teacher.}

A question has been asked regarding why we have Dhamma talks with most
of the participants passive, when we could all be engaged in discussion
or dialogue on Dhamma. At this monastery we dedicate time to both
activities: quietly listening to Dhamma and constructively discussing
Dhamma.

There is a way of listening to Dhamma talks which is well-known in the
context of Asian Buddhism, but not always so well-known here. Because of
the conditioning we have received, we are familiar with engaging in a discussion
or in debate, or listening to a
lecture. Generally speaking, we are used to having our discriminative
faculties stimulated. With quiet listening, or contemplative listening,
we give our discriminating faculties a rest and exercise ‘simply
listening’; we intentionally enter a more passive, receptive mode. This
is distinctly different from the ‘doing’ mode which we know so well. It
is important to appreciate that being in a passive, receptive mode does
not mean abandoning or bypassing our discriminating potential, but it
does call for a different kind of relationship with a mind that is
usually busy agreeing and disagreeing. With contemplative listening it
is possible to be aware if the teacher says something we don’t agree
with, but for the time being ‘park’ our objection, continue listening
and then return to the objection later.

There are times when a Dhamma talk is a form of instruction \emph{about}
Buddhism, similar to a lecture. But depending on the teacher’s approach,
a Dhamma talk can also be a form of induction into a relationship with
our inner contemplative. We all have an inner contemplative who has deep
and significant concerns, but we don’t necessarily all know how to hear
what he or she is saying. When we are debating a view or an opinion, we
are required to use our discriminative faculties. In contrast, the mode
of contemplative listening requires us to relax those faculties. When we
listen deeply, beyond just the words, we can receive much more from the
teacher than merely the information that the words themselves might
impart. Potentially we can also hear the teacher’s energy, enthusiasm,
confidence, perhaps even freedom.

Disengaging from the mind which is used to picking and choosing,
agreeing and disagreeing, can feel uncomfortable at first, possibly even
unsafe. But sitting silent and still for 40 minutes probably didn’t feel
comfortable in the beginning, not to mention bowing! It is worth
exercising this ability by way of an experiment, and learning to set
aside the active, doing, discriminating mind. It is understandable that
we might feel afraid that relaxing our hold on the thinking mind could
leave us open to being brainwashed. Nobody should surrender their
analyzing mind until they feel ready to do so. But not being daring
enough to try something new leaves us vulnerable in another way. So it
is suggested that when a teacher leads a group contemplation, the best
way to benefit from what is being offered is to temporarily let go of
our conditioned preferences and rest in quiet attentiveness. To some
this perhaps sounds like suggesting they go to sleep, which of course is
not the point (although it could happen). Rather, it is about how to
make ourselves as susceptible as possible to truth, to be able to hear
beyond the mere surface appearance of the words. The~aim is to get the
whole message.

In the Buddha’s teaching known as the Discourse on Great
Blessings,\cite{mahamangala-sutta}
he speaks about the blessings that can arise from participating in
\emph{Dhammasavaṇa}, which means listening to Dhamma teachings. In temples in
Thailand on the New Moon and Full Moon Observance Days, a teacher
typically begins the talk for the occasion with, ‘Today is Dhammasavaṇa
day. Please establish your minds in a way fit for receiving these
teachings from the Buddha.’ And it is not rare to find that when a talk
is being delivering, a ritual fan is held in front of the teacher’s
face, to encourage the assembly to listen to the message being offered
and not allow themselves to be distracted by the teacher’s outer
appearance. Whether those listening like or dislike the teacher’s
appearance is irrelevant. A mind that is fit to receive teachings is a
mind that is at ease, sensitive and open to hearing something new. So
long as we are functioning on the level of liking and disliking,
agreeing and disagreeing, we remain on the surface. As with a lake that
is disturbed when winds blow over it, we miss seeing a beautiful
reflection. To truly appreciate the offering of a Dhamma talk, it helps
to be able to rest in inner stillness, fully attentive. The time for
analyzing and debating can come later.

In the Discourse on Great Blessings the Buddha also mentions the
blessings that can arise from \emph{Dhammasākacchā}, which means sharing in
Dhamma dialogue or discussion. In this case we are making use of our
analytical faculties, but for most of us they are probably well enough
developed already. To be able to put our mental acumen to one side for
the sake of being more available, more receptive, calls for a different
type of effort. It is not because we are too lazy to think or too timid
to explore that we give our active minds a rest. Nor should it be
because we hold the view that a peaceful state of mind is an end in
itself. It is because we are interested in what the mind looks like,
feels like, when it is still. Do we hear in a different way when
thinking settles? It is for the sake of being able to tune in to the
teachings on more subtle levels, with greater sensitivity, that we
temporarily renounce our critical faculties.

In many traditional Theravada monasteries there are depictions of the
Buddha’s chief disciples, Venerable Sāriputta and Venerable Moggallāna,
usually sitting to the left and right of the Buddha. Particularly in the
Burmese tradition, these disciples are depicted with their heads tilted
to one side, as it might be when listening attentively. When considering
the etymology of the Pali word for ‘disciple’, \emph{sāvaka}, it is
interesting to find that it literally means ‘one who hears’. The
disposition of a good student is that of one who listens to what the
teacher is saying. There is a misperception often found in Mahayana
teachings, whereby these venerable disciples are unfortunately referred
to as mere ‘sound-hearers’. Maybe the etymology is the source of the
misunderstanding. Literally, the translation is accurate. However, these
noble disciples could hear way beyond the words the Teacher spoke: to the
spirit, to the essence, to Dhamma.

\section{Contemplative Discernment}

There was an occasion around 1977 or '78 when I was staying in Bangkok
at a monastery called Wat Bovoranives, at the same time as Ajahn Chah
was staying just outside Bangkok. A senior Western monk who usually
lived in a monastery in Australia was visiting Wat Bovoranives at the
time. He was particularly keen to pay his respects to Ajahn Chah, and if
possible hear a Dhamma talk. Fortunately, we were able to make our way
out to where Ajahn Chah was staying near Don Mueang International
Airport. A small group of other people had gathered there that evening,
and Ajahn Chah did agree to give a Dhamma talk. As I recall, he started
with the usual encouragement to ‘establish your minds in a fitting mode
for receiving these teachings.’ But before he went into the body of his
talk, he elaborated on what is meant by ‘a mind that is fit to receive
teachings’. Those were the days before mobile phones and MP3 recorders,
but somebody had placed a tape recorder in front of him. Referring to
this recording machine, he said that listening to teachings was like
turning on the tape recorder; we should simply trust that the machine
will do its work.

Once we turn to inner quiet, we should trust that what we are ready to
receive will be received, rest in open-hearted receptivity and allow the
peaceful heart to do its work. At some later period when the need
arises, the teachings will be there, stored away in the heart. It is not
necessary to try and understand or remember what is being said. In fact,
all the trying can get in the way. To give up trying does not mean
giving up making effort. We are learning to make a different kind of
effort. We are alert, we are attentive, we are not abandoning
discernment. But this kind of discernment doesn’t disturb our serenity.
It is not the same as listening to a lecture, where we are concerned
with accumulating information, or as debating or discussing a point of
view. This is contemplative discernment.

As with so many aspects of the spiritual journey, we learn as we go
along. In the beginning we didn’t know how to sit and walk in
meditation, but we learnt. We didn’t know how to hold our precepts in a
way that was neither repressive nor heedlessly following our habits, but
we learnt. Likewise, we can learn to listen from a place of stillness
that is attentive and interested but doesn’t disturb the calm. I heard
recently of an experiment conducted on a group of artists, where the
participants were asked how they went about appreciating a work of art.
Several of them reported that they simply took it all in as one piece;
they simply ‘received’ it. The artists were then filmed close up, with
the camera focusing on their eye movements as they looked at the work.
Although some of them thought they were appreciating it in an open
receptive mode, just taking it all in, in fact their eyes could be seen
to be darting all over the place, selecting various aspects to focus on.
Scientists have reported that our eyes are regularly making selections
at three times per second.

When we understand this, we can appreciate the tradition of gently
closing the eyes when listening to a Dhamma talk. Going inwards does not
have to be a gesture of denial of the relevance of human relations. Nor
does it have to mean that we are trying to escape into our heads, where
we might feel more comfortable. The invitation for us to close our eyes
and listen fully to what the teacher is saying, is a way of entering our
inner temple, the place to which we take our deepest concerns, our heart
concerns. If we have the opportunity to hear teachings from those who
wish us to find freedom from suffering and are willing to share their
understanding, then of course we want to receive their offering. If we
can learn to listen deeply to our outer teachers, perhaps we will meet
our inner teacher. Probably we will find that all true teachers tell us
the same thing: hold carefully, but don’t cling to anything or any view.
If we get that message, it will indeed be a great blessing.

Thank you very much for your attention.
