\chapterNote{%
Alert to the needs of the journey,\\
those on the path of awareness,\\
like swans, glide on,\\
leaving behind their former resting places.

Dhammapada 91}

\chapterTitleVertical{K\\E\\E\\P\par M\\O\\V\\I\\N\\G\par O\\N}

\chapter{Keep Moving On}

\tocChapterNote{Dhammapada 91, letting go, faith, Heartwood sutta, integrity,
fear, desire, honesty, consistency, ascetic practices, addictions, agility.}

Let us consider together this short teaching from the Buddha.

When we begin on this journey we are all seeking, in some form or
another, an increased sense of freedom. Maybe we were motivated by an
altruistic vision of living with greater compassion. Or perhaps it was
the clarity and detail of the Buddha’s analysis of the human mind that
inspired us. For many it was maybe just a matter of trying to find
relief from the burden of suffering. Whatever it was that brought us to
the beginning of this path, we all benefit from the encouragement
offered by those who have travelled ahead of us.

Here, in this Dhammapada verse 91\cite{dhammapada-aruno},
the Buddha is pointing out the
benefit of cultivating a willingness to keep beginning again in our
practice. He doesn’t want us to settle anywhere short of realizing the
goal of freedom. He wants us to keep practising whatever happens.

When we start out we can’t know what the journey will be like. There
will be periods of gladness and possibly periods of sadness. Sometimes
we will feel as if we are making progress, and at other times we might
feel thoroughly stuck. At times we will feel confident, and at other
times we might feel as if we are sinking in a swamp of doubt. The Buddha
encourages us not to settle anywhere, but to keep letting go until the
wisdom that knows the way to freedom becomes perfectly clear.

\section{Training To See With Insight}

Fortunately for us, throughout the centuries this Theravada Buddhist
tradition has maintained tried and tested teachings on the cultivation
of letting go in the pursuit of wisdom. The teachers and the tradition
are still readily available to offer us guidance. And there will
definitely be times in our training when we need that guidance.

There is a word in the Pali language that we recite in our chanting when
describing the qualities of the Buddha: \emph{lokavidu}, which means ‘one who
knows the world’. ‘Knowing’ here refers to a direct, transformative kind
of knowing. It is not the knowing with which we are generally familiar,
a knowing \emph{about} things. This path of practice is characterised not by
what we learn about on it, but by the effort we make to develop a fresh,
new way of seeing.

When we say \emph{Dhammaṃ saraṇaṃ gacchāmi}, ‘I go for refuge to the Dhamma’,
we are saying that more than anything else we are interested in seeing
the reality of the world. Here the world is not just the outer world,
but the inner worlds too, all that we create in our minds. Being a
knower of the world means knowing the truth of gladness and sadness,
confidence and doubt, liking and disliking – and being able to accord
with this truth; being at peace with it. There is no limit to the
information we can accumulate about the world, but now we are learning
to see what the world looks like when we have let go of habits of
clinging. We are training to see with insight.

\section{Not Believing}

There will be times when we feel love and gratitude towards our teachers
and the tradition, and probably there will be times when we don’t. I
spent the early years of my monastic training, most of the period
between 1976 and 1979, at Wat Pah Nanachat in NE Thailand. I recall
first arriving there and feeling filled with gratitude; there was a
tremendous sense of relief. At last I was somewhere I truly wanted to
be. I didn’t want to be anywhere else in this whole wide world, and I
felt privileged to be received into that community. The monastery had
only been established for a few months. To say food and accommodation
were basic is an understatement, but the grass-roof huts and the modest
one meal a day were just wonderful. To be in the company of the radiant
and gracious Ajahn Sumedho and his fellow Western wayfarers was a source
of joy and inspiration.

I can also vividly recall wondering months later what it would be like
if only I could escape from this hell-hole. One morning, as we walked on
alms-round and crossed over some railway tracks, I stopped for a few
moments to feel the tracks beneath my bare feet, and imagined how those
railway lines went all the way to Bangkok, and how perhaps there I could
be relieved from the intense misery I was having to endure. It can be
difficult to release ourselves from the momentum we have generated by
following the habit of believing in our moods, including both agreeable
and disagreeable moods. Thankfully on that occasion in my training I
didn’t completely believe in what the mood was telling me, otherwise I
would have given up.

Whether we love our teachers and our tradition or not, what training in
insight is about is learning to let go of the way things appear to be,
to stop merely believing and see that which is true. When we see
accurately, we arrive at a true appreciation of whatever there is in
front of us, be it our teachers, our moods or anything else. We won’t
have to be driven by conditioned liking and disliking, and that means we
won’t get stuck. All this conditioned activity of liking and disliking
is what is meant by the world. Insight sees through the world, revealing
its instability and the fruitlessness of our habits of clinging. Wisdom
sees how the clinging creates resistance and causes suffering. The
warm-hearted expression of such wisdom is compassion.

\section{Becoming Stuck}

The Buddha doesn’t want us to settle too soon, but to keep moving on
until we arrive at real wisdom and compassion. It is because of
unawareness regarding the truth of our preferences – our liking and
disliking – that our view of the world is distorted. Having preferences
is not wrong per se; it is the way we have them which makes the
difference. The untrained mind is regularly fooled by the way things
appear to be. The Awakened Ones are never fooled, hence they never
suffer.

Before his enlightenment the Buddha-to-be was also lost in liking and
disliking, he too was fooled by the way the world appeared to be and
suffered accordingly. When he felt happy he believed ‘I am happy’. When
he felt sad, he believed ‘I am sad’. But eventually, growing tired of
the mediocrity of such an existence, he set out in search of a solution.
He trained his inner spiritual faculties to the point where eventually
he could see beyond all the conditioned activity of mind, beyond the
‘world’, to what he referred to as the unconditioned. From that point
onward he knew directly that whatever conditions arose in his mind, they
were simply passing through. Nothing could get stuck. He was fully freed
and fully available to bring benefit into the world.

The Awakened Ones view all existence according to what is real.
Unawakened beings see according to what we project onto the world by way
of our preferences. Every time we attach to something, we become stuck
there for a while. If we are training rightly, we gradually learn to
recognize sooner what we are doing. If we want to measure our progress
in practice, it should be in terms of how long it takes us to remember
that we are doing the clinging, we are doing the suffering; it is not
something external that is happening to us. Whether gross or refined,
the same principle applies.

\section{The Discourse on the Heartwood}

One reason for emphasizing the importance of establishing practice on
the principle of beginning again, over and over, is because the
temptation to settle, to get stuck, can arise at any stage. We don’t
only need to be careful about our coarse fluctuating moods. It is also
possible to become stuck in refined levels of concentration.

My first meditation teacher in Thailand, Ajahn Thate\cite{ajahn-thate}
was very adept
at abiding in highly refined states of samādhi. He relates how he spent
ten years stuck in unproductive states of tranquillity. It took the
penetrative insight and helpful support of Ajahn Mun to guide him away
from his fondness for samādhi and, by establishing his meditation in
body contemplation, to proceed towards awakening.

Some of you will be familiar with the discourse by the Buddha known as
the \emph{Mahā-sāropama Sutta}\cite{mahasaropama-sutta},
the Discourse on the Simile of the Heartwood.
In this teaching the Buddha likens someone setting out in
pursuit of awakening to someone going in search of heartwood, the most
valuable portion of a tree. Initially spiritual aspirants trust that
reaching the goal is possible and are energized into making an effort.
However, quite quickly they find that just being on the spiritual
journey means they gain increased respect; their status in society
rises, and they decide this level of increased well-being is good enough
and cease making efforts. The Buddha likens this to someone setting out
in search of the heartwood but settling for a bunch of twigs. In other
words, if we find ourselves feeling pleased with the praise we receive,
for having impressed a few friends with our spiritual efforts, that is
not the place to get comfortable.

The discourse goes on by likening the aspirant who settles for the level
of elevated ease and contentment which comes with upgraded integrity to
someone settling for a portion of the outer bark of a tree. Then the
seeker who grows comfortable with the increased well-being which comes
with concentration and tranquillity is likened to someone going away
with a portion of the inner bark. Resting at the level of initial
insight is likened to the seeker becoming contented with a portion of
the sapwood. It is not until full awakening is reached that the Buddha
says the seeker has arrived at the heartwood.

\section{Beginning Again}

At any stage of practice we can be fooled into believing that ‘this is
good enough’ and abandon making efforts. We manage the risk of this
happening in advance by cultivating the wholesome habit of willingly
beginning again. This doesn’t mean we never rest or pause to delight in
the increased sense of freedom which comes from letting go. Certainly,
taking all the time we need to regularly refresh and renew our body and
mind is skilful – so long as a pause doesn’t turn into a fixed position.
The pleasure that comes with receiving praise and popularity, for
instance, can be intoxicating. Or perhaps the more subtle pleasure that
comes from samādhi could tempt you to settle. Maybe you feel it’s time
to start sharing your wisdom and compassion with the world, and set up
your own YouTube channel. But if you feel it is ‘my’ wisdom and
compassion, it would definitely be better to ‘keep moving on, leaving
behind former resting places’.

And it is not only our own increased ability that might distract us from
the path; we could become blinded by somebody else’s aura. There are
many teachers around looking for disciples, and if they catch us in
their spotlight we can lose perspective. Allowing ourselves to become
overly impressed by stories about the magic powers and super-abilities
of others, however noble they are, does not necessary bring benefit. As
the Buddha advised in
the \emph{Mahā-Maṅgala Sutta}\cite{mahamangala-sutta},
we can learn from ‘association with the wise’, but if we are truly learning, we will keep
letting go.

\section{Clumsy Beginnings}

Our ability to keep moving on is not always going to feel comfortable.
We won’t automatically start out with an ability to glide on smoothly.
Especially early on, our excessive enthusiasm can cause our efforts to
be somewhat clumsy. When I was living under Ajahn Chah, there was an
occasion when I was called upon to translate for a newly arrived novice.
This eager young man wanted Ajahn Chah’s advice on how he should set up
his practice during the approaching Rains Retreat (\emph{vassa}). He
explained that he was determined to practise really hard and intended to
take on several of the ascetic practices
(\emph{dhutaṅga vaṭṭa})\cite{dhutanga}.
He listed all the various practices he was aiming at adopting. Ajahn Chah
listened until I had finished translating, and then advised, ‘What I
recommend you should do is determine to keep practising regardless of
what happens. No need to do anything special.’

On another occasion Ajahn Chah most helpfully instructed, ‘There is
absolutely nothing to be afraid of, so long as you are not caught up in
desire.’ Wanting to make progress can feel normal. Longing for
understanding can seem perfectly appropriate. But if we haven’t really
studied closely the actuality of desire, apparently virtuous motivations
might in fact be fixed positions. It takes some subtlety to see the
truth of the matter, beyond the way wanting appears to be. If it is true
that we are not caught up in desire, there will be no fear. If we are
still concerned about having special experiences, perhaps it is because
we are being fooled by the ‘apparent’ nature of desire.

The truth of desire is that it is a movement in the mind. It is not who
we are, though we readily make a sense of self out of it. We feel happy
and think we ‘are’ good when wholesome desires pass through the mind, or
we feel guilty and believe we ‘are’ bad when there are unwholesome
desires. On closer inspection, these desires can be seen simply as
activity taking place. These movements only define who we are when we
decide that is so.

\section{Increased Honesty}

Rather than special practices which tempt us to look for special
results, it is increased honesty which is more likely to prevent us from
settling too soon. Whenever we become attached, we get stuck. It might
be attachment to our teachers, to the tradition, to techniques or to the
results of practice. But wherever and whenever we cling, we are in
effect betraying our aspiration for freedom; in a way we are lying to
ourselves. Conversely, every time we make the effort to see through the
stories that our mind tells us, to see beyond conditioned liking and
disliking, we grow in honesty. Incremental increases in honesty are a
more reliable measure of the value of our effort than whether or not we
are having special experiences.

Our teachers, the tradition, the techniques, are all approximations.
They are like maps to which, if we are wise, we will learn to relate.
Fixating on the map, no matter how impressive it might be, is missing
the point. If we are walking in the Swiss Alps and focus on the stunning
precision and detail of the map, we could fail to see the ice beneath
our feet and slip, seriously hurting ourselves. The map won’t
necessarily show us where the ice is, or if there is an angry mountain
goat about to attack and knock us over a cliff.

If we are being honest with ourselves, we admit to the part we play in
creating the suffering in our lives. We admit that we are the ones doing
the clinging; it is not happening to us. We acknowledge that although
all beings experience pain, suffering is something extra that we add to
it. The Buddha and all the realized beings experienced pain, but they
didn’t suffer. Every time we allow awareness to constrict around an
activity of mind, we impose the perception of being limited; that is, we
suffer. We are responsible for this. When we are busy looking for
results in practice, we risk not seeing what it is that we are doing and
then believing that if we are suffering it is someone else’s fault.
Likewise, if we attach too much value to books we have read or
meditation techniques, we run the risk of missing the truth which is in
front of us. When we are suffering, the truth is that here and now we
are imposing limitations on awareness. If we are honest we won’t blame
others, we won’t blame the world. And we won’t blame ourselves either;
instead we will investigate. This image the Buddha has given of
swans continually moving on, leaving behind their former resting places,
helps serve the cultivation of such honest investigation.

And when we are honest, here and now, we will be careful about the risks
we do take. One of life’s lessons is that when we have acquired a new
skill, we then need to refine that skill. It’s like learning to ride a
bike: in the beginning we have someone holding on behind, but eventually
they let go and we can manage on our own. Even if we fall off a few
times, at last we learn. Once we have a feeling for the increased
ability that riding the bicycle gives us, perhaps at first we get a
little carried away and even hurt ourselves, before arriving at a level
of competence and safety. Hopefully we don’t get too badly hurt, but
experimenting is normal.

The spiritual journey does indeed involve daring, and we need to know
that there is heedful, helpful daring, and heedless, harmful daring. If
our effort in practice is smooth and constant, we can rely on our
intuition to tell us whether or not daring is safe and appropriate. If
we listen carefully to what our teachers share from their experience,
that will help protect us from hubris. And we can trust that our
commitment to keeping precepts will also protect us and indicate when it
is safe to venture into territory where we don’t feel familiar. If
intuition is informed by modesty and is not an expression of deluded
ambition, our daring is less likely to be heedless.

Our commitment to simple honesty gives us a frame of reference. We can
trust that impulses to attach and become lost in ambition will show up
on the radar before it is too late. On those occasions when we miss the
signs and do get caught in clinging, honesty means we will own up to our
part in creating the suffering that follows, which in turn means we are
best placed to learn the lesson.

\section{Addictions}

The agility which accompanies simple here-and-now honesty shows us
where and when we are hanging onto false levels of security, where and
when we are lying to ourselves. It can also help us prepare for the
unexpected. Much of this spiritual journey involves meeting the
unexpected. We can’t know how or when awareness will reveal our
attachments; those places where we hold to fixed positions. And not just
fixed positions, but also when we are feeding on praise or popularity,
like the person setting out in search of heartwood and settling for a
bunch of twigs. Our relationship to power is similar. As years pass by,
don’t be surprised if you discover you are not as equanimous towards
power as perhaps you thought you were.

We might also have to look again at something as basic as our
relationship to food. Take sugar. It took me over 40 years as a monk
before I really got a handle on sugar. These days I refer to it as
low-grade heroin and stay well away from it. I regret that I couldn’t
own up sooner to what was behind my addiction to sugar.

\section{Consistency}

If our effort in practice is consistent and the emphasis is on letting
go rather than achieving, we will be in the optimal position to own up
to attachments when it is the time to do so. Whether attachments
manifest as an insensitivity to how we relate to power, or as addiction
to a false source of energy like praise or sugar, or perhaps a subtle
identification with some long-standing unacknowledged personal
‘problem’, they can all be met and let go of. And it certainly makes a
difference if we have prepared ourselves in advance with a conscious
willingness to keep moving on, however good or bad things might appear.

If we start out from a place of confusion and insecurity, we might feel
tempted to settle for a modest degree of increased confidence. Or if we
have had to work very hard in our practice, perhaps we feel tired of
making an effort and want to give up. But even wanting to give up can be
acknowledged and let go of. Wanting to give up doesn’t mean we have to
give up. When we are able to see desire as a movement in mind, this
means the desire is ready to be received and released. Don’t assume it
defines who we are. Being able to see it is one of the fruits of
practice.

Our teachers have shown us what agility looks like, and how it is
possible to live without fixed positions. We are most fortunate to have
the example of their lives. Regardless of how likeable or dislikeable
any experience might be, our task as students of the way is to have the
honesty and daring to turn the light of attention around and to face the
experience, to see it for what it is, and keep moving on.

Thank you for your attention.

