\chapterTitleVertical{T\\H\\E\par A\\R\\T\par O\\F\par M\\E\\D\\I\\T\\A\\T\\I\\O\\N}

\chapter{The Art Of Meditation}

\tocChapterNote{Vipassanā, scientific research, techniques, gentleness, hope,
art, agility, repetition, Middle Way, source-orientation, walking meditation,
creativity, innovation.}

Some of you may have come across the scientific articles around these
days extolling the benefits of meditation. Research into the effects
that meditation may have on the brain has produced evidence of
considerable benefits. I have also recently come across articles
disparaging and discouraging Buddhist meditation. These are written by
people who have tried but after a while given up, claiming it is
dangerous and even life-destroying. And this is not just because they
haven’t tried hard enough; tourists who perhaps did one \emph{vipassanā}
retreat while they were in India and then gave up. Sometimes these were
people who had been applying themselves for years to meditation, but
still ended up disillusioned. This is very unfortunate. However, I
confess I’m not altogether surprised. Having been the leader of a
monastery for over twenty years, I have heard from a lot of people about
the various approaches to meditation practice.

\section{Initial Approach}

When we first come across these teachings, we are presented with not
just another belief system, but with something we can actually \emph{do}
about our consciousness, and this gives us hope. So we enter the path
with enthusiasm, confidence and energy. We throw ourselves into
developing the technique; and perhaps we get some results. But what do
we do next? Once we’ve had some experience, especially some sort of
‘special’ experience, it’s easy to cling to the memory. If it was
pleasant we may try to repeat it. If it wasn’t pleasant we may still
cling to the memory, afraid that it may be repeated.

It seems to me that sometimes there is a risk that the way meditation is
taught over-emphasizes the technique. If we cling to a technique, we
tend to also cling to results – leading to just more clinging, even
though the Buddha was very clear that clinging is the cause of
suffering.

In the beginning we do need techniques and we can learn from them. But
the idea that they are all there is to meditation is regrettable. It
took me quite a while to realize that a technician’s approach wasn’t
working. I eventually came to see how preoccupied I had become with the
\emph{idea} of meditation. The point of the practice, the spirit, is to
deepen in understanding and ease. Always worrying about whether or not
we are doing the technique right – about stages to pass and skills to
develop – can easily lead to narrow-mindedness. Then the next thing we
discover is that our efforts in the spiritual life have only served to
make us dull. Naturally, we sense something important is missing.

When we over-emphasize the techniques, we readily forget the place of
gentleness. Considerable sensitivity is needed to see the underlying
attitudes that lead to suffering. Holding too firmly to the forms of
practice can cause insensitivity and rigidity. For instance, if I have
the attitude that there was something wrong with me and that these
techniques are going to fix me, my effort can make me feel more limited
than before. This is because our heart-energy is going into feeding the
gaining mind, the idea of never being good enough and always having to
get somewhere.

How we pick up the techniques determines how we relate to experience in
general. Particularly in the West with our strong, wilful attitude to
life, relating to meditation as a self-improvement technique is very
pronounced. We need to be careful that we don’t bring our wilful
manipulative tendencies into spiritual practice – which is surely the
most important aspect of our lives. Good health, meaningful
relationships, money, food and shelter are all significant, but when we
die the most important thing will be our state of mind. So the way we
enter our inner exploration is most important, and we shouldn’t think
that everything is explained in a manual.

\section{Meditation As Art}

These days I find the contemplative life is more usefully viewed as an
artistic exercise or as a craft. In the beginning we need to learn the
skills involved, as we would with any art form, like playing a musical
instrument, for example. Initially, applying ourselves to the techniques
can be boring, but becoming adept requires repetition. To play a violin
well we must learn how to move our fingers, how to hold the wrist. If we
hold the instrument incorrectly, we miss out on many beautiful
possibilities. Hours and hours of exercise are required in the beginning
to learn to play an instrument, or to use the medium of paint or handle
a camera. However, once we’ve internalized the skills required, once
they’ve really become ours, we can let the spirit of the artist flow. I
suggest it’s similar with meditation.

If you do not see yourself as particularly artistic, you could see it in
terms of agility. One of Ajahn Chah’s teachers used to advise: if
obstructions appear high, duck under them, if they appear low, jump over
them. The approach of ‘one size fits all’ can get in the way of our
imagination. If we feel we must adhere solely to what a beloved teacher
taught us when we first started out, we could stop learning, and as a
result lack the creativity to deal with the obstacles encountered along
the way. Unfailing respect and gratitude to those who helped us get
started, yes; but also dare to go into the unknown, and with interest in
discovering something new. The Buddha tried several different teachers
before he finished his work. If he had stayed with the first teacher and
meditation method that he found, he might never have become enlightened.

\section{Middle Way}

So perhaps the authors of those commentaries on the perils of meditation
hadn’t felt allowed to experiment in their practice. Maybe they felt
practice was all about one single technique. Because a famous teacher
tells us what we should be doing, that does not mean they necessarily
know what we truly need. What is needed is to locate the in-between
ground, where we sincerely and respectfully listen to and apply
ourselves to the teachings we have fortunately been offered, while at
the same time listening to ourselves. That is the middle way: not
grasping at ‘my’ way of doing things, and not grasping at the teacher’s
way of doing things either, but studying both.

Early on in practice I definitely tried to follow what my teachers told
me and had some delightful, peaceful experiences, particularly resulting
from concentrating on the breath. But did they really help me deal with
the obstructions which I had to face? Only up to a point, and then they
failed miserably. I suspect this happens to a lot of spiritual seekers.
Meditators get to a point where they feel they’re banging their heads
against a brick wall.

I would like to encourage listening more carefully to what our own
intuition has to tell us. See it as developing a friendship, in the way
we would get to know and deeply care about somebody else. Certainly it
is suitable to observe the experiences of others. We can learn from
looking and listening outside to books, teachers, traditions, but we
should also pay careful attention to what comes from inside. I am not
advocating grasping at an idea that ‘my’ personal, unique and amazing
approach is absolutely the way, but let’s not assume either that our
intuition has nothing to contribute. What matters is that we are growing
in confidence and commitment as we progress on this journey.

\section{Where Energy Comes From}

It matters that we feel allowed to include creativity in our meditation.
Generally speaking, our education has encouraged us to use lateral
thinking to deal with issues. It has taught us to approach things
creatively. If now, in the spiritual domain, we are told we are not
allowed to do that, and the tradition insists we adhere solely to one
venerable approach, that can stifle interest. And it is definitely not
helpful to lose interest. Interest is one way of interpreting what is
meant by the Pali word \emph{viriya.} Viriya is one of the five spiritual
faculties\cite{faculties} and it is essential to spiritual development. Usually this
word is translated as energy, but where do we get our energy from?

As I have mentioned elsewhere, for those who find confidence in letting
go of their gaining mind, who benefit from what I call a source-oriented
approach, wilful striving doesn’t work. On the other hand, sustained
interest in present moment awareness, and trusting in letting go of all
goal-oriented striving, nurtures enthusiasm in practice, it generates
energy. For source-oriented types, a sense of being creatively engaged
in our relationship with our meditation is essential.

\section{Unlocking Practice}

On my very first seven-day meditation retreat the teacher taught
\emph{ānāpānasati}, mindfulness of breathing while sitting; he also taught
walking meditation. I remember how on the third day of this retreat I
had a wonderful experience, a sudden perception of inner peace. There
was a quality of inner quietness like nothing I’d experienced before. I
was out in the countryside, walking up and down on a gravel road in a
remote part of NSW, Australia. With this perception of peacefulness came
an inner voice – the chatterbox who likes to have an opinion about
everything – commenting, ‘There’s just awareness’, or perhaps it was,
‘There’s just knowing.’ Then a question spontaneously arose, ‘But who’s
aware?’ At that point the mind dropped into a deeper, even lovelier
place. I can’t recall exactly how I reported this experience to the
retreat teacher, but he didn’t seem to appreciate it as a useful key for
unlocking my practice. It took a long time and a lot of struggle before
I appreciated it for what it was.

Conscious questioning as a form of meditation is nothing new. Lots of
people use it as a way of disciplining attention and exploring the inner
terrain. Asking the right question, your own heart-question, can be a
powerful part of practice. Such questioning is not coming from the head.
There are times when concentrating on a meditation object can be a
pleasant, agreeable thing to do, but maybe we should view it in the way
Ajahn Thate\cite{ajahn-thate} taught his monks.
He used to tell his disciples that
entering samādhi was what monks did instead of going on holiday. He
would encourage it. However, going on holiday is not the work.

Some of the most interesting work I do is asking questions like, ‘Who’s
aware?’ I happen to also enjoy thinking about such things as the
architectural plans for developing this monastery, but the more valuable
work is asking deep questions like, ‘Who is asking this question?’
That’s an extremely interesting question – if it’s asked in the right
way, and not because I or somebody else told you to ask it.

Our mind might be longing to ask its important questions. Regrettably,
many people approach the activity of their mind as an enemy. All they
want to do is make their mind shut up, so they concentrate, concentrate,
concentrate, in pursuit of peace. There are other aspects to this
exercise besides developing concentration. Maybe you can make your mind
your friend. Your friend the mind might really want to share this
journey with you and have some valuable contributions to make.

There are spiritual traditions in which teachers specifically encourage
asking questions. Asked in the right way, at the right time, in the
right direction, our heart-question can be the very thing that begins to
tease out the tangled threads of our contracted heart. At one stage in
his practice when Ajahn Fan, a disciple of Ajahn Mun, was caught up in
fear, he went to consult with his teacher. Ajahn Mun didn’t just say,
‘Go and concentrate on your breath and make your mind peaceful.’ He
asked Ajahn Fan, ‘Who is it who is afraid?’ Master Hsu Yun\cite{hsu-yun}, the
great Chinese Chan meditation master, used the technique of asking
‘Who?’, called in Chinese \emph{hua-tou}, the profound question practice.

\section{Asking In The Right Way}

Remember, these ‘pointings’ to the way are not to be grasped. If we
cling to them with the gaining mind, they will once again be deluded ego
building itself another shelter. Be careful not to grasp at this idea of
asking the question, ‘Who?’

It is not the activity of our minds which creates the idea that there is
a problem; it is the deluded notion which expresses itself as
self-centredness. That’s the issue; much of our energy is being consumed
by this construction. So how can we release that energy, how do we undo
it? As we have said, certainly there is a stage when learning to bring
the mind to one-pointedness, to steadiness, is needed. But that’s only
one part of our training; can we take it all the way? Not necessarily,
not everybody. Some people may take that form of concentration
meditation nearly all the way; and I’m told that at the very last stage
of practice, at just the right time, they ask some very subtle questions
and the whole tangle unravels; they find the freedom they’ve been
seeking. But that may not be the way for all of us. Indeed, I suspect
it’s not the way for many of us. Maybe we need to trust that our mind is
not our enemy and make friends with it, learn to listen to it.

Followers of the Christian tradition teach, ‘Ask and ye shall be given.’
When I was a Christian I used to ask all the time, but I didn’t get the
results I was looking for. Only years later did I meet a Christian monk
who pointed out that it matters how we ask. If we’re not asking from the
right place we’re not going to get the right answer.

If we are fortunate and persist on our inner journey, we might come
across our own personal question, the one that will untangle us; but we
need to be careful about how we ask our important question. Our
questions need to be accompanied by a humble recognition that we don’t
know. In my first year of meditation, when I was applying this
questioning practice, there were periods when I was using it like a
sledgehammer. That didn’t work well. It didn’t help at all, actually; I
became very sick. I have some photographs of what I looked like then;
they’re frightening! We need to ask our questions gently, respectfully,
as if we were having a conversation with someone we look up to.

\section{In What Is All This Taking Place?}

Related to this, I often reflect on a question Ajahn Chah once asked.
It is recorded in the introduction to the book, Seeing the Way, Volume
2\cite{seeing-vol2}. A group of young monks were talking with him about the Original
Mind. He pointed out that they must be very careful not to make this
Original Mind into a ‘thing’; if they did, that was not the Original
Mind. If there’s anything there at all, he said, just throw it all out.
You can refer to an Original Mind if you want to, but the concept,
‘Original Mind,’ is not what is being pointed to. He went on to point
out that what is truly original is inherently pure; there’s nothing you
can say about it. If you do want to discuss it, words are necessary, but
don’t get caught in the words.

In the course of that conversation Ajahn Chah asked the question, ‘In
what is all this arising and ceasing?’ You can be watching arising and
ceasing all the time, \emph{but in what is it all taking place?} That is a
powerful question. We can be following some meditation technique,
observing arising and ceasing, arising and ceasing, but be so caught up
in applying the form of the meditation exercise that we forget our own
organic interest in being free from suffering. So Ajahn Chah’s asking
where or in what it is happening is a helpful tool for getting us
unhooked from the technique. All the arising and ceasing is happening in
awareness, knowingness, the one who knows or whatever we choose to call
it. It requires a shift in perspective to see the context and let go of
focusing on the activity. Whatever word we use, of course that’s not it.

\section{Creative Involvement}

Carefulness and creativity go together. I learnt one technique aimed at
bringing us back to mindfulness in the moment from the teacher Ruth
Denison. It involves having people stand on one leg. I have sometimes
used it, even when talking on the telephone to someone lost in
confusion: ‘OK, come on, let’s both get up and stand on one leg.’ Maybe
they think I’m kidding: ‘I’m serious. We’ll talk about your problem, but
right now, let’s stand on one leg. If you want to talk to me, we’ve got
to be standing on one leg first.’ So there we are each in the middle of
a room, with the telephone at one ear, standing on one leg. That’s a
very useful exercise, because to do it we have to let go of thinking and
come back into the body. After we’ve stood on one leg for a while, old
habits are likely to draw attention back into the head; but then we’ll
wobble, and when we’re about to fall over we have to come back quickly
into the body. Maybe they tell me, ‘But I can’t think about my problem
while I’m standing on one leg!’ To which I reply, ‘Well, that’s good,
because that’s why you rang me, because you couldn’t stop thinking about
your problem.’

I’m not being flippant when I talk like this; the exercise is useful if
you find yourself lost. You can even do it in public situations so long
as you are discreet and nobody notices! And again, we’re not talking
about grasping the technique and becoming one of those Indian ascetics
who stand all day on one leg. I suspect they’ve missed the point.

There are lots of techniques that we can employ to train our attention.
Ajahn Chah wouldn’t allow electricity in the monastery for many years;
he insisted we pull water from the well by hand. I expect he saw that as
a good way of embodying mindfulness practice. It also worked well in
training monks to cooperate. I was recently speaking to the monks here
in our monastery about a Zen temple where the abbot wouldn’t allow a
washing machine, concerned that the students would become lazy.
Eventually the monastery did acquire a washing machine, so the abbot
said, ‘OK, when you put your clothes in the washing machine you must sit
and watch the washing go round and round in a circle. You may not just
push the button and go away and get heedless again, you’ve got to sit
there.’

Ajahn Chah banned cigarette smoking at his monastery, but when I first
ordained I lived in a monastery in Bangkok where it was still allowed.
But the rule there was that you weren’t allowed to smoke unless you were
sitting down, so if you were going to smoke you had to smoke fully. Of
course, I’m not advocating that particular practice. But the message
being conveyed, the spirit that was in effect encoded in that structure,
was to do what you’re doing fully. If you’re writing an email, fully
write the email. Often when we are sitting at a computer, we are lost.
We forget the body and become stressed. We’re not really doing what
we’re doing. We are not quite all there. Yet we’ve heard our teachers
say over and over that the practice of mindfulness is here and now. The
Buddha said, ‘The past is dead, the future’s not yet born.’ The only
reality we have access to is this reality, here, now. We benefit from
having structures that effectively help bring ourselves back to this
moment. But let’s remember that the structures are not an end in
themselves.

So if the way you already use a meditation technique nourishes your
faith and strengthens your confidence, do continue. If a more flexible
approach appeals to you, if you feel drawn to a somewhat more creative
involvement in your meditation, don’t automatically reject that feeling.
It might be your mind coming to help you on the journey.
