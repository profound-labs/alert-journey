\chapterNote{%
  ``Maybe we are not yet confident about what we are trusting in, but that is
  perfectly fine; we are cultivating a whole new attitude towards life and that
  takes time.''}

\chapter{Confidently Not Knowing}

\chapterTitleVertical{C\\O\\N\\F\\I\\D\\E\\N\\T\\L\\Y\par N\\O\\T\par K\\N\\O\\W\\I\\N\\G}

\tocChapterNote{Dying, shift in perception, disorientation, paradox, identity,
source-orientation, near-drowning, trust, refuge, precepts, renunciation, sense
restraint.}

Somebody has asked this question, ‘What would be a good death?’ Let us
contemplate this together.

When I read this question, the thought that comes to my mind is, ‘I
really don’t know what a good death is’. Obviously our death and how it
happens to us is significant. But I can’t say I am persuaded by the
temptation to speculate about whether or not I will die well; or even
what ‘well’ might mean. What I do find interesting, though, is how I can
be fully present for the way my body and mind react to this question.
Right now, what I know is that I don’t know; the truth that can be known
at this point in time is that I am not sure how my death will unfold.

Over recent years I have had the privilege of being with a number of
people as they have approached death, and have witnessed some very
beautiful, peaceful passings. We all know that death doesn’t always
happen like that; there are those who struggle terribly as the end
approaches. With this in mind, I could become preoccupied with concerns
about how death might happen in my case. Perhaps I’ll fail at it. After
all these years of practice, who knows what’s going to happen? But what
benefit would such speculation bring?

There is another option available. We can cultivate the ability to
receive such a question into a quality of awareness which is
intentionally established ‘here-and-now’; which includes the whole
body-mind and is free from compulsive tendencies to take sides for or
against whatever might be happening. When we receive such a question
into this kind of awareness, something very different occurs. Usually we
contract due to fear when confronted with the state of
not-knowing; usually we have a distinct preference for the feeling of
knowing. But if we train to confidently tolerate the perception of
not-knowing, such a question becomes a precious teacher. It can show us
how to truly let go.

If we choose the path of cultivating awareness, it will hopefully not be
long before we discover that security and safety are to be found in an
attitude of trusting, not in following our habitual efforts to wilfully
control life. Our addictions to controlling only draw us deeper into the
currents of craving. The desperate attempts to manage uncertainty never
end. We’re all familiar with feeling caught up in wanting to be sure,
afraid of being unsure, wanting to be in charge. We know how much
suffering that can generate. This teaching is an invitation to train in
trusting in awareness itself, rather than blindly following conditioned
desires for certainty.

\section{Currents of Craving}

Fortunately, we have access to a tradition of teachings which focuses
specifically on recognizing and then skilfully letting go of these
currents of craving. At the very core of the Buddha’s teachings there is
an emphasis on letting go. This is not a spiritual journey aimed at
accumulating; it is about release and surrender. Enthusiasm and vitality
manifest along the way insofar as we open up to the utter uncertainty of
existence, not to the degree our attention contracts around a set of
beliefs. This is not a teaching about becoming more certain. Rather, we
are reminded to come back over and over again to the feeling of
uncertainty in the whole body-mind and really get to feel it, not merely
replace a disagreeable feeling of uncertainty with a more agreeable but
synthetic sense of certainty. At some point we have to learn to be
completely OK with knowing that we don’t know.

Anyone familiar with the teachings of Ajahn Chah and Ajahn Sumedho will
be aware of how frequently they encourage us to cultivate conscious
letting go. On the first occasion of my meeting Ajahn Sumedho, I was
struck by the simple, but beautiful way in which he was able to say no
to a second cup of coffee. That sounds like a small and insignificant
thing, but it left a vivid and meaningful impression on me. We had
enjoyed an initial cup together, and then his attendant offered him a
second cup. Somehow, he seemed able to say ‘No’ in a way that I had
never witnessed before. His manner wasn’t that of a self-conscious
somebody, doing something special to get somewhere, which was probably
what I would have expected from those living the religious life. It was
a plain and simple ‘No, thank you.’ It was new and delightful to meet
someone with both a sense of humour and clear discipline. I had known
people who were fun to be with but not particularly principled. And I
had known those who were seriously disciplined, but certainly not much
fun. Here was someone who appeared able to honour a commitment to
spiritual training, but without denying life. Here was the result of
wise cultivation. Later, when I met Ajahn Chah, I found that he too had
both an infectious laughter and an unshakeable but completely
uncompromising commitment to discipline.

Of course, saying no to a second cup of coffee is not the ultimate goal
for which we strive, but if spiritual training doesn’t include all
aspects of life, it is not really worth following. I will always be
grateful to Ajahn Sumedho for showing me that it is possible to address
these currents of craving and be able to let go in a thoroughly natural
manner.

\section{Relief and Disorientation}

When we give ourselves into this training, a sense of relief may dawn as
we realize that the struggle of being someone trying to get somewhere is
falling away. As we learn how to let go, as we learn how to hold our
goals more skilfully, more accurately, the journey takes on a very
different characteristic; the terrain through which we travel feels as
if it is opening up, the air becomes more pleasant to breathe. A
trusting heart is beginning to replace deluded ego’s dramatic efforts to
control. Maybe we are not yet confident about what we are trusting in,
but that is perfectly fine; we are cultivating a whole new attitude
towards life and that takes time. Entering this territory can feel
exciting and invigorating. It might also feel slightly dangerous.
Indeed, it is sensible to proceed with caution.

Not everybody experiences this new emerging perspective with gladness
and relief. For some it brings increased confusion, possibly on many
levels simultaneously. When letting go is the priority in practice, it
will eventually lead to a falling away of the old familiar sense of
identity which was born out of clinging to mental images of being
someone who knows who they are and where they are going. We probably
hoped that dropping those old identities would lead to boundless bliss
and profound understanding. But for some, what emerges instead is
disorientation; extreme feelings of anxiety and of being threatened. If
we are not properly prepared, as these old worn-out identities start to
fall away and we encounter the unknown, fear triggers a contraction of
awareness. At the very moment when what is called for is a trusting
attitude – opening our hearts and minds to the wonderful discovery of
something new and freeing – our awareness collapses into an agonizing,
even horrifying, state of doubt. If such a collapse occurs, the task of
familiarizing ourselves with conscious not-knowing takes on even greater
significance.

\section{Seismic Shift}

For some, practice is a series of incremental episodes of letting go,
leading more or less smoothly to a gradual deepening and increased
clarity. For others it is more a case of surviving a totally unexpected
fundamental shift in perception, a seismic sort of letting go. These
people suddenly find themselves seeing in a way that they have never
seen before. That can feel like being relocated to a completely new
abiding which they sense to be the \emph{context} of all experience, that in
which all activity is seen as \emph{content}. From this perspective, whatever
arises and ceases is effortlessly recognized as the way reality has
always been and always will be, only there is no longer the familiar
relationship with a sense of ‘me’. Any sense of self, agreeable or
disagreeable, is perceived simply as activity, essentially irrelevant to
the seeing, to just knowing.

We are not talking here about another kind of experience, but about a
shift in the relationship to all experiencing. It can appear as
perfectly normal, nothing special, but also as profoundly important and
utterly transformative. Such an opening might occur in the context of
formal practice, or it might happen spontaneously without any apparent
preparation. For some who see in this way, such an opening up persists;
the clarity remains forever just so. For others, consciousness seems to
revert back into finding identity as that which is in motion: as
content, as struggle. In this latter case, what remains is the memory of
a powerful shift having taken place. Of course, this memory is not the
same reality as what is being remembered, and it can take considerable
skill to avoid conflating the two.

However, even when direct access to the pure, just-knowing mind is no
longer evident, a powerful intuitive understanding of deeper dimensions
of mind remains. A different quality of insight is now accessible. For
example, it can make perfectly good sense that there is one kind of
activity manifesting at the surface level of the mind, while at the same
time another totally different activity is manifesting at a deeper
level. One type of desire might be occurring at one level, while at the
same time exactly the opposite desire is happening at another. From the
perspective of just knowing – from the perspective of awareness itself –
there need be no conflict in any of this. This insight makes it much
easier to tolerate otherwise apparently contradictory states of mind. In
fact, such frustration can now serve as a dynamo generating energy, not
something to be viewed as an obstruction.

When we find refuge in awareness itself – and this is a sensitive
feeling-awareness, not something abstract and disembodied – apparently
conflicting mind-states do not have to be viewed negatively. The
apparent conflict provides the fuel that takes our investigations
deeper. Feeling conflicted doesn’t have to equate with having problems.
Ambiguities and uncertainties can serve as an impetus to let go more
deeply, encouraging us to trust more thoroughly. We only struggle when
we become lost in identifying \emph{as} the content of awareness. Awareness
itself never struggles. And by receiving into this awareness questions
which challenge us and take us to uncertainty, we can now pay attention
to what intuition is telling us, instead of just being informed by
superficial conditioned thinking.

Once again though, we do need to be careful. The very same fuel that has
the potential to ignite insight also has the power to cause burn-out.

\section{Losing Control}

These Dhamma teachings that have been handed down to us are like a
reservoir of understanding into which we have the good fortune to be
able to tap. We live at a time when individuals who have walked the path
ahead of us and can offer hints on how to train ourselves are available.
If we have the patience and agility of attention to hear and heed their
counsel, that can save us a lot of trouble. When I think about the
appropriate attitude towards our teachers and this tradition, the word
that comes to my mind is ‘reverential’.

Just because you may have touched into deep stillness, that doesn’t mean
the momentum of mental and emotional habits will immediately cease and
you will find yourself thoroughly transformed. It doesn’t mean you
haven’t still got a lot of work to do. Remember, the deluded ego always
loves to be in control. Losing control is the last thing that any
deluded ego wants. And since our addictions to delusion have previously
been so well fed, the powerful currents of ‘me’ and ‘my way’ are not
likely to quietly fade away just because a radically new perspective on
reality has revealed itself.

\section{Addiction to Understanding}

These conditioned currents of craving express themselves in many
different ways. For instance, if letting go has opened you up to a
radically new perspective, it is very likely you will feel you just
\emph{have} to understand it. This is a good time to consider that perhaps
you are being pulled into the current of craving for knowledge \emph{about}
reality. We want to ‘get it’. From the perspective of the old identity,
and given the kind of conditioning to which we have been subjected, such
an impulse appears perfectly justified. Following the desire to know –
read that as ‘control’ – is how we have made our way through life thus
far. However, from the perspective of training to trust in the
just-knowing mind, in awareness itself, this needs to change.

A commitment to letting go of the craving to control means surrendering
ourselves, over and again, increasingly fully, into simply knowing that
which can be known here and now. And as we said at the beginning, when
the truth is that we don’t know, right practice means knowing just that
much, knowing that we don’t know. Let’s not attempt to push past that
feeling of uncertainty just because it is uncomfortable. It has
something important to teach us. If it happens that a totally new way of
perceiving reality has manifested, let’s not rush to secure our old
sense of self by grasping for a conceptual understanding \emph{about} it. It
is not necessary to understand ‘this’, even when ‘this’ seems profoundly
new. The same principle applies when your heart has opened to something
wonderful and radiant but which then passes. Once again craving to ‘get
it’ is likely to occur; this time we are trying to get the new
perspective back again. The clarity perhaps appeared so genuine and felt
like the most authentic you have ever been, but now it has passed.

Allowing ourselves to be caught in desire won’t help; it will only lead
to more struggle. What does help is knowing what we can know, here and
now. Learning how to make just the right kind of adjustments to effort
in such situations requires great subtlety, sensitivity, humility and
patience.

\section{Fine-tuning the Enquiry}

\enlargethispage*{\baselineskip}

Becoming caught in old patterns of compulsively attempting to make
ourselves secure with conceptual security is not the same thing as
developing contemplative enquiry. We have a natural and wholesome
impulse to understand, conducive to calm and deepening. If our impulse
towards understanding means we are still struggling to prop up the old
sense of ‘me’, this struggle will disturb the peace of mind that we need
for investigation. Contemplative enquiry is more a matter of attuning
ourselves to the reality that is in front of us; it is not struggling to
‘get’ something; it is more like making ourselves available.

And the types of questions that we ask in this process of fine-tuning
are important, as is how we ask them. You might try preparing yourself
for this level of subtle enquiry by imagining you are sitting in front
of the Buddha. He has granted you an interview. You have your burning
question; how would you ask it? Presumably not in a demanding way.
Probably not in a casual way. You have interest, energy, perhaps
tremendous energy, and of course you have respect. The way in which we
approach our enquiry makes a difference.

As your practice of letting go proceeds, be prepared for surprises;
including the surprise of coming across old emotional content which
needs revisiting. Even after years of meditation and hours of therapy,
you can still find you have emotional content that is not fully
received, not yet fully let go of.

If we are in too much of a hurry to get over the apparent obstructions
that we encounter, we run the risk of compounding issues. It is more
useful to slow down, learn to receive these apparent ‘obstructions’ and
work on a willingness to accept them as they are. Everything we
encounter on this journey, both the agreeable and the disagreeable, has
something to teach us. Judging what we meet as right or wrong doesn’t
help. Regardless of how embarrassing or humiliating the contents of our
minds might be, what is called for is an increased capacity to simply
receive them~all.

Just because we encounter a mind state that we haven’t read about in the
\emph{suttas}, that doesn’t mean it’s wrong. These states are only wrong if
we make them so. Mind states arise dependent on causes. However raw and
unattractive the contents of our minds might be, what matters is whether
we react in ways that lead to more clinging and compounding of
suffering, or expand the field of awareness, accommodate the conflict
and arrive at letting go.

\section{Dissolving Identity}

The spiritual exercises that our teachers give us are specifically
designed to dissolve the armour we have constructed around life’s pain.
Potentized awareness is supposed to dilute the deluding effects of
personality belief. Our commitment to personality only became
established in the first place as a defence against the suffering of
life. Now that we have better tools to work with, we can approach life
directly, with all its joys and sorrows, and give up manipulation. We
can embrace suffering, welcome it, bow down to it, invite it to teach us
what we need to know about reality, and then let it go.

Deep insight does have the effect of stripping away the armour, but what
is revealed may not be what we expected. Radiant and uplifting though an
open, trusting heart may be, the resulting increased sensitivity can
leave us feeling intensely vulnerable. Perhaps we start doubting, and
the question arises, ‘How could so much fear follow from so much beauty?
How come I feel so ungrounded and threatened after feeling so utterly,
effortlessly secure?’ Hence the encouragement to prepare ourselves for
not knowing, for absolutely anything: mental disruption, emotional
challenges, weight loss, weight gain, relational upheaval. Perhaps we
meet individuals with whom we feel we can share as we have never shared
with anyone before. Or maybe we meet people we wish we had never had to
meet.

Too much thinking about how the path should unfold or too much comparing
of ourselves with others, just feeds resistance. Undoing this tangle of
self-belief is always unique. There has only ever been one of us. But
there are patterns and similarities, which is why heeding the guidance
given by those ahead of us on the path is skilful.

\section{Source-Oriented Practice}

The mystery of how the path will unfold for each of us, including how
our death occurs, is something for which we can train ourselves with
conscious, intentional trusting – trusting in that which is already here
behind the habits of resistance, behind the armour of
personality-belief.

I have often spoken about source-oriented practice and how it contrasts
with goal-oriented practice. Depending on how they have been
conditioned, some individuals benefit from having a clearly articulated
sense of a goal ‘out there’ to strive towards. For others this approach
is a luxury they can’t afford; such an approach means they lose touch
with the ‘actuality’ of this moment. For those who find a
source-oriented approach to practice makes more sense, it matters that
they feel allowed to relax their hold on any ideas of a goal out there;
their emphasis needs to be on expanding awareness so as to accommodate
more fully, more willingly, whatever is happening, here and now.

Relaxing a hold on ideas of the goal is like relaxing your shoulders
when driving a car. It doesn’t mean you let go of the steering wheel or
never look at the sat-nav. Source-oriented practice engages the ability
to trust and receive, while goal-oriented practice will perhaps give
more emphasis to doing and achieving. Generally, those of a
source-oriented persuasion are less intimidated by diversity and paradox
and can take practice into any situation, while goal-oriented
practitioners seem to benefit from stability and predictability, and
might be less comfortable with complexity.

When it comes to contemplating death, source-oriented practice means
paying close attention to any impulse to control the process, not taking
a position against the mind’s habitual desire to control, but not
indulging in it either; simply trusting in the power of the just-knowing
mind. This is not grasping at a belief in the idea of trusting, or
trying to convince ourselves that it is the true way. Rather, we are
looking at what happens when we let go of our attempts to control and
choose to intentionally trust. By way of contrast, we can study what
happens when we engage the judging mind, speculating about how it should
or shouldn’t be. We feel our awareness contracting and release out of
it; feel in the whole body-mind, what that release feels like, and see
how much more accommodating it is. We feel how the resistance, the
suffering, fades. We see how intuitive intelligence becomes available in
open-hearted, trusting awareness, and how it is compromised when we
contract and cling. If fear happens, we study fear. Fear of failure for
instance, is not failure, unless we say it is. It is simply a movement
in awareness that is ready to be received. If you are able to abide \emph{as}
awareness – \emph{as} just knowing – there need be no struggle. We don’t have
to struggle to get anything right or fix anything when we are not
identified \emph{as} that ‘anything’, \emph{as} the activity.

\section{Be Careful Who You Talk To}

Between source-oriented and goal-oriented practice, it is not that one
way is right and the other wrong, but that they are different, just as
people are different. And they are not mutually exclusive. Especially in
the early stages of practice, we can experiment with aspects of both
modes. It is useful, however, to know what works in our own case. It is
useful to understand what kind of effort is needed in any given
situation. It also helps to be careful whom we talk with about our
practice.

Goal-oriented practitioners might consider emphasizing the cultivation
of a trusting attitude as heedlessness and argue that the Buddha taught
to strive on with diligence. Indeed, we all agree that the Buddha did
teach striving on with diligence. But just what diligence looks like is
another matter. If a turtle tried to explain to a fish what it was like
to walk along the beach, and how lovely it could be to soak up the warm
sunshine before returning to the cool ocean, the fish might think the
turtle had a problem, was hallucinating. Of course, in fact the fish
doesn’t have access to the same reality as the turtle. Each one’s
perception is valid, but a turtle should be careful about trying to make
a fish understand the attractions of leaving the water. Somebody told me
recently about conversations they read in online chat rooms, saying
that Ajahn Chah didn’t know how to practise and had it all wrong.
Certainly those whose primary inspiration comes from reading books,
reading \emph{about} Dhamma, could get confused by what teachers like Ajahn
Chah have to say about the Buddhist path.

\section{Preference for Certainty}

It is not just followers of theistic religions who look for certainty in
how they hold to beliefs. When fear causes a contraction of our
awareness, it is probably because we are caught in desire for certainty.
Hence our teachers’ encouragement to contemplate uncertainty –
\emph{aniccaṃ}. Despite all the encouragement, however, many followers of the
Buddhist path still grasp at a conceptual understanding of the teachings
on impermanence in order to feel secure. Or they grasp at meditation
techniques, including those techniques specifically geared to lead to
insight into impermanence, to try to give rise to a feeling of
certainty.

Just how we relate to the teachings and the tradition is something we
must get to know for ourselves. Whether our confidence in this path of
practice is truly dependable or not is revealed whenever we feel
challenged. Do we revert to habits of propping up the sense of being
someone, doing something, to get somewhere? Or do we surrender; open,
receive, let go? This could include letting go of the sense of being
right – being willing to lose an argument, for instance. Right practice
never means propping up or promoting the feeling of ‘me’.

We can rely on our sense of confidence if we find we are able to welcome
suffering when it appears. This doesn’t mean that we like suffering or
would wish it upon ourselves or another. But, as the Buddha taught, it
is mindfulness of suffering that leads to freedom from suffering. How
willing and able are we to simply receive suffering?

\section{Suffering as Pointing}

The impulse to resist and reject suffering might appear to run deep. It
is not easy to feel sad or afraid without assuming we are somehow
failing. But so long as we still perceive suffering as an indictment of
our progress on the spiritual journey, and we believe that perception,
we are undermining ourselves. In truth, any time we suffer, to any
degree, we are receiving teachings. One evening early on in my monastic
training, when we were all sitting in the main meeting hall at Wat Pah
Pong, Ajahn Chah ascended the Dhamma seat and began his Dhamma talk by
saying, ‘Don’t feel bad if you are suffering. We all suffer.’ I remember
being surprised and relieved at the same time. That I was surprised
suggests his words conflicted with some view I was holding about
practice. This spiritual training is not about trying to avoid
suffering. Trying not to suffer is like trying not to wake up in the
morning because you prefer to dream. Both sleeping and waking are
natural for human beings, and so is experiencing both pleasure and pain.
What matters is how we accord with this. How accurately do we perceive
that which we experience? This is different from a life committed to
following preferences.

Last night at evening puja we chanted the Buddha’s discourse called
The Turning of the Wheel of the Law, or the \emph{Dhammacakkappavattana
Sutta}. In this discourse the Buddha explained how to skilfully attune
to the reality of the world we live in, all of it, with all its pleasure
and pain, its agreeability and disagreeability. The Buddha’s Great
Awakening was the realization that clinging to anything at all – any
possession, any view, any practice – eventually leads to suffering.
Attuning to reality or finding refuge in Dhamma means studying suffering
until we get the message and experience letting go. When we try to be
someone who doesn’t suffer, we strengthen the habits of clinging and in
the process we create more suffering. Indeed trying to be anyone at all
means we are still caught. If we understand this point we can become
interested in suffering instead of merely rejecting it. We can become
interested in refining our quality of attention, of patience, of
kindness, so we can recognize the reality of whatever life gives us and
not allow ourselves to be fooled by the way life appears.

\section{Vortices of Craving}

In the process of studying life, whether it be in our daily-life
practice or through developing formal meditation, we gradually learn the
skills required to recognize the signs which indicate we are about to
get caught up in desire. If we don’t catch ourselves before we cling,
but only find ourselves once we are already born again into being
someone, doing something to get somewhere, that is the time to
re-establish awareness. That is where we learn. No judgement! When we do
find ourselves being dragged down by the vortices of craving, it doesn’t
help to indulge in judging ourselves for having become lost. Fighting
doesn’t help either. Nor does mental proliferation about why it
shouldn’t be this way. What can help is remembering our here-and-now,
whole body-mind awareness, and trusting.

Some years ago I was swimming off the west coast of the North Island of
New Zealand, near a place called Piha. It’s a particular part of the
coast well known for good surf and dangerous rip currents. There I
experienced vortices of a different but equally threatening kind. Having
been a strong swimmer when I was young, it didn’t occur to me that I was
putting myself in danger by swimming there. A friend and I had been
hiking for several hours along the coastal footpath, and since the beach
below us was empty, it seemed fine to cool off in the water. What I
didn’t notice was that at the point where I chose to enter the water,
the waves were not breaking. Had I been better informed about the nature
of breaking waves, I would have recognized the absence of white-water
breakers as a sign that there was probably a hollow area in the sand
beneath the surface of the water, creating a counter-current that could
pull anyone or anything that entered there out to sea, and being pulled
out to sea is exactly what happened to me. Many drownings result from
just such situations, when a swimmer is unexpectedly caught in a rip
current and reacts by struggling desperately against it until exhaustion
eventually takes over. Initially I definitely struggled, trying to get
back to the shore and out of the danger, doing what I was used to doing
whenever I felt threatened, trying to save myself. But I realized quite
quickly that no amount of fighting to overcome the current was going to
work; it was far too strong. What did work, thankfully, was
surrendering; I flipped over onto my back and floated; no more fighting,
but simply allowing the current to carry me.

Just prior to this incident I had been introduced to a particular
breathing technique that involved deep relaxation, deep trusting and a
whole-body surrendering of habitual controlling. Somehow in that moment
of intensity I remembered what I had learned and found myself drifting
out to sea, floating and breathing. My head was filled with powerful
conflicting thoughts and images: of being eaten by sharks somewhere
between Piha and Sydney; of my parents being upset on hearing that their
son had drowned; of Ajahn Sumedho being annoyed with me for my
heedlessness. But at one point, associated with the effort to keep
floating, trusting and breathing, came the powerful thought, ‘Let the
Buddha take over’, my translation of \emph{Buddhaṃ saraṇaṃ gacchāmi} – ‘I go
for refuge to the Buddha’.

It felt like a battle going on within me, between on the one hand strong
inclinations towards trying to save myself, and on the other an impulse
towards trusting. The thought that I mustn’t give up the struggle to
save myself was fuelled by guilt and distrust, and when I engaged it,
the rhythm of the breathing was interrupted and my body began to sink.
When there was letting go of the contraction of fear and trusting again,
the body felt held and supported and I returned to floating. There was
no doubt about the intensity of fear coursing through my body; I
definitely did not know that I was going to be OK. At times it really
did look like I might not be. Thankfully, the intimidation of the
‘not-knowing’ state was outshone by the impulse to surrender into the
breathing, to trust, to releasing out of the struggle to save myself. I
didn’t drown.

As it happened, the current did drag me out to sea quite a way, but then
carried me down the coast, out of the dangerous area, and eventually the
waves brought me safely ashore. Once I was standing on the beach again I
felt elated: not just because I was now safe, but because I felt I had
been given the gift of affirmation of practice. In a modest but
significant way, it felt emblematic of what it meant when the Buddha
conquered Mara. I am obviously grateful that I was already equipped with
some skill in how to meet the state of not knowing before finding myself
in that threatening situation.

\section{Silent Contemplation}

Just reading about what it means to Go for Refuge is not enough. Just
studying \emph{about} reality is not enough. We need to refine our enquiry,
which means bringing all our sensitivity, all our interest to bear on
what life gives us, in daily life and formal practice. In the beginning
we read about what the Buddha taught. That can give us a good feeling,
some increased conceptual clarity about why we suffer so much. But as we
progress, as our investigations deepen, we find that conceptual
understanding only takes us so far. We need to find out what it means to
investigate without thinking. What does contemplation in silence sound
like? What does feeling investigation, contrasted to conceptual
investigation, feel like?

For example, when faced with an upthrust of fear, intense ill-will or
passionate indignation, do we lose ourselves in it and lose our ability
to reflect in the process? Or can we meet it, accord with it, and if
necessary ‘ride’ this current of energy until it subsides, and thereby
avoid drowning in it? So long as we remain committed to controlling
life, we run the risk of being overwhelmed by it, of drowning. But this
suggestion that we might sometimes have to ride the energy of these
currents is not to say we should follow them and ‘act out’. Because of
our commitment to the basic moral precepts, we have a sense of safety
and are able to experiment with investigating what it is like to face
the unknown. Our sense of safety doesn’t come from not daring to step
outside of what is familiar and comfortable. When we are facing death,
it is not likely to feel familiar or comfortable. However, if we have
cultivated awareness to be here-and-now, to include the whole body-mind
and be free from compulsively taking sides, perhaps we will find we have
the willingness and readiness to meet the unknown with open-heartedness
and gratitude.

At least at this point in time, when there is no indication of imminent
death, this seems like a practical way of approaching the matter of
thinking about dying.

Thank you for your attention.
