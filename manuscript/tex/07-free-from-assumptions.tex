\chapterNote{%
Enlightenment can happen whether sitting, standing, walking or lying
down. Some people think a lot, and when they sit in meditation they are
not peaceful, but through contemplation of happiness and suffering they
can still come to know truth.

Ajahn Chah}

\chapterTitleVertical{F\\R\\E\\E\par F\\R\\O\\M\par A\\S\\S\\U\\M\\P\\T\\I\\O\\N\\S}

\chapter{Free from Assumptions}

\tocChapterNote{Underlying assumptions, contemplation, beliefs, faith, self
view, insights, rock stars, inner terrain, discipline, 1960's, consumption,
personal responsibility.}

Readers of the Buddha’s discourses will have regularly come across words
like tethered, bound, chained,
fettered.\cite{dhp-276}
Such words point powerfully to feelings of limitation, of being constrained. Since the
Buddha wanted us to feel encouraged on our spiritual journey, why did he
use negative terms so evocative of painful feelings? Because truly wise
beings know that only when we see that we are the ones responsible for
creating such perceptions of limitation, will we realize that we are
also the ones with the power to stop creating them. So long as we still
believe that external conditions are to blame for our suffering, we keep
looking outside for something or someone to save us. Having spent
several years searching outside, the Buddha himself found that such a
pursuit was pointless; it only serves to obscure our potential for
awakening.

On occasions when a more positive message was called for, the Buddha
also spoke about the state of imperturbable peace which he had realized,
an irreversible state of inherent well-being that is effortlessly
sustainable. And he made it clear that the possibility of achieving this
state of well-being is available to all those who cultivate their hearts
and learn to see clearly. However, over and again he came back to
pointing to that state with which we are all already familiar, the
experience of constricted being, of limitation; in the Pali language,
\emph{dukkha}.

\section{What We Assume to Be True}

There is one particular form of limitation to which most of us are
prone, and which I would like to pick up as a theme for contemplation. I
am speaking about the habit of holding blindly to assumptions about
reality. In many cases it is these uninspected assumptions which are the
very cause of our living in a painful state of perpetual contraction, of
fear. The task of learning to see clearly how and when we are holding to
these troublesome assumptions is a truly important one.

The short extract above from a translation of teachings by Ajahn Chah,
points to assumptions some people hold about how enlightenment happens.
There are many dedicated meditators who cling to an idea that the
attainment of enlightenment depends entirely on sitting meditation. We
can spend a lot of energy trying to become enlightened, but if we
haven’t yet seen how we cling to such underlying assumptions, our
efforts will sadly fail us. Instead of attaining an increased ability to
live our lives fully – both the joyous and the challenging aspects – we
end up more limited. Rather than coming to a softening, opening,
trusting relationship with life through our commitment to the spiritual
exercises, we find ourselves feeling more obstructed.

Besides commenting on how enlightenment happens, Ajahn Chah might also
have been referring to the views people often have about what
constitutes ‘real practice’. There are many who cling passionately to
the assumption that doing the real practice means spending long hours
sitting in a painful cross-legged posture – as if going for a walk
couldn’t be a form of meditation. The Buddha and many of the awakened
disciples used walking meditation as a regular form of practice. Ajahn
Chah is pointing out here that all our activity can be included in
practice: formal sitting, taking a walk, talking with a friend, cooking
or writing.

When we are not careful, uninspected assumptions about reality increase
hindrances on the path. If we have looked closely at our views about
practice and decided that we should spend more time sitting, that is
fine. We are then in a position to take responsibility for the results
of our actions. It is when we are not aware of what motivates us that we
risk being caught in assumptions about ourselves, each other and the
world, and cause unnecessary extra suffering. Instead of doing our own
practice, we might easily get lost in trying to become like somebody
else, imitating what worked for them. For practice to really ‘work’, we
need to know that we are the ones responsible for creating this tangle
of confusion and be deeply interested in untangling it, instead of
looking for someone else to take responsibility. All of us – those just
starting out and those who have been meditating for many years – are
therefore called to keep looking, ever more closely, at any views to
which we might be clinging, and find effective ways of letting go of
them.

\section{Life's Questions}

This teaching by Ajahn Chah is a gift particularly appreciated by those
who find it a struggle to make their minds peaceful in formal sitting
meditation. He is telling us that we don’t have to feel guilty if we
can’t make our minds settle when we sit. Sitting meditation works well
for some people; for others it might be more appropriate to travel this
journey in a different sort of vehicle, and he refers here to wise
contemplation. If we endlessly struggle to make our mind peaceful by
focusing on the end of the nose, but the mind refuses to become
peaceful, it is fine to carefully turn attention around and look
directly at the unpeaceful activity itself. Instead of trying to stop
thinking, skilfully direct the thinking towards those concerns which are
genuinely relevant to you. For instance, why is it that we keep creating
a problem out of happiness and suffering? Why do we keep making problems
out of life? The Buddha and the great disciples lived in the same world
as we do, they encountered difficulties just as we do, but they didn’t
suffer. What are we doing that makes the difference?

Whatever our way of life might be, whether as a monk or nun in a
monastery, or a householder, we are all regularly confronted with life’s
deep questions. We can’t escape those questions. We can ignore them, but
they are still there in our hearts. We can try very hard to have only
happiness, but we will still end up in a conflict between happiness and
suffering. Failing to receive our heart’s questions into awareness can
push feelings of conflict further into unawareness. And when conflicting
conditions persist for a long time, they can give rise to another one of
those underlying assumptions; that is, that there is something really
wrong with me because I am struggling. We might be somewhat aware of
such a negative self-view, but nevertheless keep fighting to overcome it
by concentrating on the meditation object, hoping the view will go away.
If we find that such effort is fuelling the conflict, we need another
approach. When we shine the light of wise contemplation directly onto
the assumption that there is something inherently wrong with me, perhaps
we will discover that it is the way we are holding views which creates
the perception of having problems.

We are all conditioned with views about our being better, equal or worse
than others, and trying to eradicate all such views is likely to be
fruitless. On the other hand, changing the way we relate to those views
can be transformative. Contemplating our views, listening to them,
learning not to automatically judge them, helps relax our grasp on them.
Then we realize that we can trust ourselves to find our own way out of
suffering. It is a great relief when we come to see how we are actually
doing the clinging, and that we can choose to stop doing it. While
sitting meditation and trying to make the mind peaceful might not have
produced such an insight, maybe a more contemplative approach will work.
Genuine confidence arises with such a recognition.

I have spoken elsewhere about developing the skill of contemplation, but
it bears repeating. Applying contemplative enquiry is not the same as
following the everyday undisciplined thinking mind. If we are
contemplating a question such as ‘Why do I continually set up happiness
and suffering against each other?’ we are aware not just of the question
that is being asked, but also of the space around the question, the
silence out of which the question arises. We can stop and start the
thinking process; we are not being dragged along by it. In contemplation
we can intentionally ask a question and then feel what happens. Just as
we might cast a pebble into a lake and watch as the ripples spread out,
we can feel what happens when we drop our important question into whole
body-mind awareness. But to be able to truly receive our whole body-mind
response entails inhibiting any habits of disappearing back up into our
heads in search of conceptual understanding. The response which
interests us in contemplation is one that is felt, not something that is
figured out. With compulsive, conditioned thinking, the way we usually
ask ourselves questions leads to internal arguing back and forth between
views, until the mind feels drained of energy. Contemplative enquiry, on
the other hand, generates energy and ease.

\section{Inwardly Illiterate}

Recently I have been considering why it was that back in the 60’s and
70’s, so many of us suffered so intensely as we embarked on our
spiritual search. I can see now that at the time I was really
spiritually illiterate. I could read books, but I couldn’t read
inwardly. I couldn’t see the views to which I was clinging or what was
motivating me, for example, the self-view that gives rise to a sense of
entitlement. I still find it embarrassing to think about how I
approached things. Sadly, many of us didn’t even survive to be able to
reflect on our uninspected attitudes. I recognize now the naivety that
shaped many of us. We had read books like Paul Reps' \emph{Zen Flesh, Zen
Bones}, or Alan Watts' \emph{The Way Of Zen}; we had been wowed by pictures
of Tibetan thangkas; we had heard about what can happen when you practice
Theravadan meditation techniques; but how did we approach these
venerable traditions and techniques? A lot of us merely assumed that it
was going to be easy, and that all we had to do was turn up at the
monastery and get the goodies. It was a type of spiritual plundering,
like our ancestors who occupied Burma and came away with bejewelled
Buddha images to take back to their homes in Britain, but in our case we
were looking to ‘come away with’ enlightenment. No wonder we suffered so
much. Approaching spiritual masters who had spent their lives dedicated
to purifying their hearts from defilements, and assuming that we were
entitled to take whatever we could get, was a perfect way of developing
disappointment. We were compulsive consumers, but didn’t see it.

Some teachers misread the zeal of their young Western aspirants and
perceived our intense energy as spiritual ability. They had never
witnessed anyone suffering from such a desperate condition of deep
alienation and acute unhappiness. In some monasteries Westerners were
given special conditions; they might be excused from joining in with
work projects to repair the meditation huts, for instance. Or perhaps
they were allowed to skip chanting because they were so serious about
their meditation practice, and were given a special dispensation. Ajahn
Chah was not at all impressed with any expectations we had about being
special, and gave out no such dispensations.

There was one particularly notable occasion when a Western visitor to
Wat Nanachat requested an interview with the abbot. He presented himself
to the teacher and began by boldly asking, ‘So, what is it Wat Nanachat
has to offer me?’ This fellow was unaware of his conditioning as a
compulsive consumer and sadly allowed a whole set of unhelpful
assumptions to get in his way.

\section{Rampant Consumerism}

The vast majority of cultures around the world these days accept the
assumption that to ‘acquire more’ is to somehow increase our personal
value – to consume is good. I did a Google search on ‘rampant
consumerism’ and came across a paper called \emph{The Gospel of
Consumption}.\cite{gospel}
The article described intentional efforts in the early part of the
last century to create a society which was in a state or permanent
dissatisfaction, so that producers would have an limitless market for
their products. As a result of acceptance by society of such
ill-considered social conditioning, most people now drive themselves in
endless pursuit of gratification, consuming all they can. They
eventually abandon any hope of genuine contentment and, like lemmings,
dive off the cliff into an ocean of complacent mediocrity. Because of
the collective agreement to collude in this way, most of society is now
occupied in slaving away to acquire the means to consume more: material
goods, services, information. This blind habitual behaviour becomes a
disposition, so that almost everyone ends up caught in constantly
consuming. Even sleep is spoken of as something that we either do or
don’t ‘get’ enough of. Behind this particular disposition is the belief
that one day we will have an experience or acquire a possession which
will give us the satisfaction we believe we are lacking. But this
consumerist attitude of needing to perpetually acquire more is nothing
more than being blindly caught in the vortex of deluded desire.

If we contemplate what is really happening here, and instead of merely
following desires apply mindfulness, sense restraint and wise
reflection, maybe we will find that we can look craving directly in the
face - and then discover that at least for a moment, we have already let
go of craving and realized that there was nothing lacking from the
beginning. The consumerist attitude is a con. The unexamined assumption
of the validity of the consumerist view is only there to keep us
dissatisfied and enslaved. With wise contemplation we can free ourselves
from this view, and see that what creates the impression of something
lacking is our heedlessly following desire. If we allow the energy of
desire to return to the source, to remain at home in the heart, instead
of always going out after objects, the sense of dissatisfaction will
cease.

\section{Questioning the Sense of Self}

A more subtle set of assumptions is to be found in how we hold the
sense of self, the sense of who we experience ourselves to be. When
asked who or what they are, many people would probably refer to their
thoughts, their emotions or possibly their bodies, or maybe a
combination of them. But if we assume our sense of self is to be found
by identifying with our thoughts, emotions and physicality, what happens
when we grow old and these things don’t function as we would wish? Does
that mean our sense of self collapses? Is suffering in old age an
obligation? That is a great question. It is one of the questions that
occurred to the Buddha-to-be at around the age of 29 and motivated him
to set out on his spiritual journey.

Towards the end of his life, as Ajahn Chah’s health was deteriorating
and his physical faculties were starting to fade, he helpfully described
what was happening for him. He spoke about knowing that he would intend
to say something like, ‘Sumedho, come here’. But when he opened his
mouth the words he heard himself say were ‘Ānando, come here’. However,
he said this didn’t disturb him in the slightest, since he knew it was
simply a matter of the physical faculties falling apart. The knowing
itself, the awareness, was undisturbed. So where was Ajahn Chah’s sense
of self located? Or maybe you think he didn’t have one!

Is our sense of self a fixed thing? Generally speaking, most of us tend
to assume so, and we invest a massive amount of time and energy in
promoting it. But which specific self do we think is permanent or real?
It doesn’t take a lot of introspection to see that there are many
‘selves’: the happy me, the unhappy me, the alert, together me, the
confused, exhausted me. Which one is really real? From a contemplative
perspective they all have their validity, yet none of them is ultimate.
Ajahn Chah had a series of strokes and his wiring became scrambled, but
his awareness remained undisturbed because he knew that none of the
conditioned activity of his mind was who he really was.

For unawakened beings, whenever our sense of self is threatened or
challenged, we suffer. For awakened beings there can be no suffering,
since they see beyond any sense of self; they know that all thoughts,
emotions and physical conditions are simply the continually changing
activity of nature. Their sense of who they are is not to be found in
the changing conditions.

If we haven’t looked deeply into the perceptions we hold about who and
what we are, we readily accept the collective assumptions fashionable at
any given time. During earlier periods of evolution, human beings seem
to have found their sense of identity in terms of the tribe they
belonged to, or in their family. More recently, identity has been found
in terms of the nation to which people feel they belong. And these days,
for many, it is sought by identifying with our individual ego
structures: personal patterns of thinking, emoting and physicality. From
a contemplative perspective, all this activity of ‘self-seeking’ can be
studied, felt, observed and, hopefully, eventually, understood as simply
conditions arising and ceasing. We gradually learn not to cling to any
of it. We keep going deeper in our questioning and enquire: in what is
all this arising and ceasing taking place? Can we sense the space out of
which all this activity appears and into which it disappears? If we
train our spiritual
faculties\cite{faculties}
in this way, there is surely a better
chance that when our physical faculties start to disintegrate, our
perspective on reality won’t disintegrate with them.

In the meantime we can use formal meditation and daily-life experiences
to investigate all these perceptions of selfhood. How ‘real’ are they?
How permanent are they? Is there a dimension of mind that is free to
witness the various ‘selves’ appearing and disappearing? What happens
when we try to find a self in the witnessing, just-knowing dimension?
Remember, these questions are an invitation to contemplate, to go
deeper, they are not questions to be answered conceptually. Following
the example of the Buddha-to-be, we embrace these questions and let them
guide us until we reach direct understanding.

The first Western woman to join our Sangha in Britain as a nun was
Sister Rocana, previously known as Pat Stoll. In a conversation about
practice with Ajahn Chah, Pat Stoll once asked the question: ‘Since the
Buddha taught \emph{anatta}, non-self, how can we practice concentration
meditation? Surely, when we are concentrating, there needs to be a sense
of somebody there doing the focusing on the meditation object.’ Ajahn
Chah’s reply was wonderfully succinct: ‘When we are developing
concentration meditation (\emph{samādhi}) we work with a sense of self. When
we are developing insight meditation (\emph{vipassanā}) we work with
non-self. And when we really know what’s what, we are beyond both self
and non-self.’ If we try to grasp conceptually what Ajahn Chah was
pointing at in this statement, we are likely just to give ourselves a
headache. This type of pointing is directed not at the head but at the
heart, at awareness itself.

\section{Self Importance}

A few years after our monastic community first moved from Thailand to
Britain, a series of discussions took place about our style of morning
and evening chanting. I remember this particularly well because I was
not included in the discussions. Various community members thought we
should take the opportunity to ‘correct’ the inaccuracies in our
pronunciation of the Pali language. Personally, I found our traditional
daily chanting an enjoyable and significant part of the monastic
routine, and had no problem with employing what is sometimes referred to
as poetic licence. There didn’t seem to me to be any need to ‘correct’
our chanting. It seemed fine that when intoning the Pali words for the
sake of recitation, we didn’t have to adhere so strictly to rules which
would quite rightly apply if speaking the Pali language.

When rumours started circulating that a new style of chanting had been
developed, I can’t say I was pleased. The truth was that I felt
thoroughly miffed that our beautiful chanting was being replaced with
something in which I had had no say. As soon as I heard the new style, I
immediately disliked it. It sounded to my ear as if it had been created
in a laboratory; much of the warmth and rhythm had been replaced with
something that a computer could have come up with.

Around the time I was pondering on how to express my disappointment, I
came across an article describing what happened in a Christian monastery
when the Normans took over Britain. One of the ways in which the new
leaders established control over the people was to replace the Saxon
abbots of the monasteries with Norman abbots. These new abbots insisted
on introducing their own style of chanting. In at least one monastery a
group of rebel monks refused to abide by the ruling and insisted on
chanting in the old style. It seems that no amount of persuasion could
make them budge. So it was decided to employ the royal archers to force
the change. As the monks gathered for chanting in the sanctuary and
commenced their ‘old’ style, the archers up in the gallery started
picking them off with arrows. Reflecting on this lesson from history, I
decided the more sensible attitude would be to let go of the assumption
that I was entitled to be consulted on everything, and accord with what
our abbot was asking.

If we develop our potential for inner enquiry and not just inner
proliferation, we find we have a valuable tool. It is a tool that we can
apply in the art of contemplation and make use of when addressing life’s
challenges. We don’t have to be so afraid of the feelings of frustration
that accompany life’s dilemmas. We are allowed not to know how to handle
a situation. We are developing the skill of holding dilemmas carefully,
sensitively, with interest and patience, and quietly waiting for a
solution to reveal itself. And when dilemmas are resolved in this
manner, it doesn’t feel that ‘I’ solved them. Humility protects the
heart from laying claim to something it doesn’t own.

It is wise not to wait until we are faced with a major dilemma before
developing this skill, but rather to build up strength gradually. Then,
if life presents us with a major dilemma, we are more likely to be able
to meet it. We might even see it as a gift instead of a disaster. On one
level we could feel as if the predicament we find ourselves in is
absolutely impossible; there is no way out! But on a deeper level, there
can be a quiet confidence telling us it is OK to feel that way; we don’t
have to act on that assumption. Very likely we really want someone else
to help us out, ‘If only …​’, yet we find ourselves all alone. Or we
feel it is up to us to make the right decision, but in all honesty can’t
be sure what the right decision is. With a well-developed ability to
hold dilemmas, we can feel the frustration and let it be; no need to
make anything out of it. Feeling frustrated is only a problem if we say
it is.

Recently I saw an interview with the frontman of a famous rock group. I
was genuinely moved by the humility that he expressed. These days this
group regularly has 60,000 adoring fans crowding into a stadium to see
them; they have been performing now for nearly 20 years. I thought back
to how Jimi Hendrix, Janis Joplin and Jim Morrison didn’t even make it
to 30; this frontman is approaching 40. In the interview he spoke about
the major dilemma he had had to face as he struggled to find his real
self. There was the self he experienced himself to be when he was on
stage for 90 minutes receiving intense adulation; and then only a few
minutes later, another self was manifest, the one he was when he was a
father with his family. What resolved the dilemma for him was learning
the skill of being able to sit with these challenging feelings of
frustration, of not knowing, and to wait and trust until awareness
opened up, and he found that it was perfectly possible to accommodate
both perceptions of self. They were both valid perceptions. There need
not be any conflict. His evident modesty confirmed that he really knew
what he was talking about. Also, his commitment to an inner life means
that he does an hour of yoga a day, fasts one day a week and avoids
alcohol and sugar. In other words, he has a committed relationship with
his inner contemplative.

\section{Assuming Nothing}

When the consequences of our past unawareness become apparent, it is
wise to welcome them. We don’t have to allow them to shape our lives. I
hope that this contemplation on uninspected assumptions means we will
stop assuming too much about anything and learn to question everything.
If we come across a ‘no-go’ area in our minds, that is a particularly
good place to spend time. Only fundamentalists countenance no-go areas.
As followers of the Buddha we are encouraged to go everywhere, to look
everywhere. And don’t be afraid that a keenness to enquire will damage
faith. An initial, uninspected sort of faith can feel threatened by our
asking our heart’s real questions, but genuine faith, reliable faith, is
strengthened by enquiry. It certainly matters that we ask in the right
way and at the right time. These important questions deserve to be
treated with respect. It would be good if we bowed down to them.

Thank you very much for your attention.
