\chapterNote{A Guided Meditation}

\chapter{Epilogue -- Be Like a Tree}

\chapterTitleVertical{B\\E\par L\\I\\K\\E\par A\par T\\R\\E\\E}

\tocChapterNote{Alignment, softening, broadening, listening, receiving, the
just-so reality.}

Those who regularly read or listen to my Dhamma talks will know that I
rarely give formal meditation instruction. There are a number of reasons
for this, not the least being that we are all so different. It is true
that there are basic principles which apply to everyone – when we cling
we create the causes for suffering; when we let go we undo those causes
– but just how we arrive at letting go differs from person to person. To
suggest that there is only one way to practise properly is to
underestimate the complexity of our human condition. And I feel it is
unkind and unwise to ignore the individual strengths and limitations of
seekers on this journey. I have experimented over the years with a
variety of formal meditation techniques, and also with various body
awareness and conscious breathing exercises. What it comes down to, in
my view, is trial and error; we try to do the best we can and learn from
our mistakes. When our actions of body and speech are guided by a
commitment to a life of integrity, hopefully the consequences of any
mistakes we do make will not be too serious. On this occasion, since it
has been specifically suggested that sharing what I have found works
could be helpful, I am happy to attempt to do so.

When I first started out on this path, the effort I was making would
best be characterized as controlling. All deluded egos love compulsive
controlling, and I was quite good at it. It suited me to hear the
teacher speak about sitting meditation as an exercise in concentrating
on the breath, and walking meditation as concentrating on the placing,
lifting, placing, lifting, of the feet. I was ready and able to apply
myself with gusto to these exercises, and I had some interesting
results. The benefit of those early efforts was evident in the
enthusiasm I felt for pursuing the practice. However, the limitations
quickly showed up when the initial delights which can come with a
beginner’s mind faded away. To progress beyond the fascinating new
perspectives which manifest when attention is concentrated required
letting go of habits of controlling. A big part of me didn’t feel so
good about letting go. I liked holding on to ‘my way’ of doing things.
But an ability to concentrate and control is not enough when it comes to
meeting the many and varied obstructions we encounter on the way. This
fits with what the Buddha teaches us about the Four Right
Efforts.\cite{right-effort}
To apply the same kind of effort, regardless of the nature of the
apparent obstruction, is not likely to be successful.

In all the different approaches to practice which I have tried, the
single most helpful meditation instruction I have received is Ajahn
Sumedho’s teachings on listening to the sound of
silence.\cite{inner-listening,aj-sumedho-sound-of-silence}
It was a relief to discover that however compulsive one’s controlling
tendencies might be, the meditation object of the sound of silence
remained constant and undisturbed. Unlike the rhythm of the body
breathing, which can become irregular if we pay attention to it in the
wrong way, the sound of silence is always there, wonderfully just so.

It turns out, however, that even this practice of listening to the sound
of silence is not always enough on its own. I have found that for many
who use this practice, it is quite possible to be paying attention to
this subtle inner sound, yet be thoroughly out of touch with the rest of
the body and the world around us. It is also possible to attend to this
background sound and remain very rigid in how we hold the overall
body-mind. Further, as a result of having been taught for years to
concentrate, pay attention and focus, many of us have ended up with a
very narrow, cramped perspective on life, our field of awareness having
collapsed. And all our efforts to fix our perceived problems can lock us
into a perpetual ‘doing’ mode, always going somewhere to get something
to make ourselves better. But addressing these symptoms of imbalance
does not have to be an onerous chore. I see it as like inheriting a big
wonderful house which is in need of refurbishing and redecorating. It
can be a lot of fun to commit yourself to such a project.

\enlargethispage*{\baselineskip}

When I look back over my years on this spiritual journey, I see six key
prompts or suggestions which have emerged as significant ‘signs’. I find
these six ‘signs’ or ‘prompts’ serve as helpful reminders and support
for an embodied presence. The recollection of these six prompts
constitutes what these days I would call my formal practice. A typical
session of sitting could involve intentionally bringing these six
prompts to mind and dwelling for a while on each one, and then,
depending on which point happens to attract particular attention,
resting there for an extended period. After I invest in these prompts
during formal practice, the mere mention of one of them in the context
of daily life can effect a helpful shift back towards balance. And this
is a practice that we can take anywhere. No special conditions are
needed.

Perhaps I should mention here that personally I have very little
interest in special experiences or special states of mind. What does
interest me is the possibility of developing a quality of awareness that
is able to meet whatever life offers. Is it possible to be buffeted by
the eight worldly winds\cite{worldly-dhammas}
 – praise and blame, gain and loss, pleasure
and pain, honour and insignificance – without being blown over by them?

The six signs or prompts are: aligning, softening, broadening, gently
listening, simply receiving and the just-so reality. So let’s look at
these in detail.

\section{Aligning}

Establishing a sense of embodied ‘alignment’ is similar to what some
people do with ‘body-scanning’ meditation, but in this case it aims
particularly at a perception of being upright and grounded. Try
experimenting with suggesting to your mind, \emph{‘Be like a tree’}. Consider
how the upper branches and leaves of a tree are reaching for the
light, while at the same time its roots are firmly planted in the
ground, and both are absorbing essential nutriment. Recollect the
Buddha’s discourse on meditation on breathing in and breathing
out,\cite{anapanasati}
where he begins by describing how the meditator \emph{goes to a quiet place,
takes a seat under a tree, sitting upright, holding the body erect,
establishes mindfulness…​} For those of us fortunate and agile enough
to be able to develop the full- or half-lotus postures, that is good.
But the rest of us may apply the principle of ‘uprightness’, to whatever
posture we are able to adopt.

Meditating on this first sign means cultivating a familiarity with a set
of specific points within the body which conduce to a sense of being
aligned. Begin by bringing attention to the area at the top of the back
of your head and feeling the sensations there. Imagine you are being
lifted up from that point. As you visualize that, also feel your chin
and see if it is being tucked in just a little. Inhibit any inclination
to make it happen intentionally by using muscles. See if imagination
alone can trigger a subtle shift, with your head neither falling forward
nor tilting back.

Now, moving down the body, bring attention to the feeling of the tip of
the tongue as it rests gently touching the roof of the mouth behind the
front teeth. Remember this is an effort to ‘align’. We are using our
imagination to direct attention to a sensation. Once you are clear that
you can really \emph{feel} the tip of your tongue, not just think about it,
go back again to the top of the back of the head, then return to the tip
of the tongue. Back and forth, slowly, gently.

Now move awareness to your shoulders. Bring to mind an image of
carrying two heavy buckets of wet sand. Feel your shoulders drop down,
way down, and allow the chest to open. When we are misaligned we easily
fall into a habit of stressing our tongue within the mouth cavity,
clenching our jaw, holding our shoulders up and cramping our chests
close, even while we are meditating! Not only do these habits compound
the state of stress, but they also waste a lot of energy.

With awareness of the sensation of the top of the back of your head,
with a feeling for the tip of the tongue gently touching, with the
shoulders relaxed and chest resting open, feel now for the weight your
body is exerting downwards onto the cushion or seat where you are
sitting. Without forcing anything, allow the body to rock very slightly
forward and then backward, and then forward again, until you find the
point of maximum downward pressure. Visualize completely flattening your
seat just by sitting on it. This is exercising ‘aligning’.

\section{Softening}

Due to unawareness, most of us grow up gradually accumulating a
backlog of unlived life. Sadly, nobody has taught us the difference
between the natural pain which all beings experience, and the suffering
which occurs as a result of our clinging to experience. As a defence
against this increasingly difficult to deal with suffering, we fabricate
forms of rigidity. If by middle age we are not already alert to these
defences, from about age 40 onwards a type of emotional rigor mortis
starts to set in, with a dispiriting insensitivity. Well before middle
age many people have already compromised their natural sensitivity, and
as a result they feel chronically obstructed when it comes to simply
feeling what they feel. Sometimes meditators wonder why, after they have
been making so much effort for so long, they are still so unhappy. Being
imprisoned behind these rigid defences against denied life might be the
cause. Softening helps with this. What we are aiming for is a softening
of attitude, but softening in the body is a practical and effective
place to begin. To cultivate conscious softening, try suggesting ‘\emph{Be
like water}’ to your mind. When you gently immerse your hand in water
there is almost no resistance. This perception of no resistance
contrasts with our habits of struggling for and against life.

Now bring awareness once more to your head; this time feel your eyes and
invite them to soften. Imagine your eyes floating gently, comfortably,
at ease, as if they have been set free from having always to be staring
at something. Feel your forehead soften, feel your jaw soften. And, very
importantly, feel your belly soften. Being soft is not being weak.
Flowing water is powerful, yet it can accord with everything it
encounters. This is exercising ‘softening’.

\section{Broadening}

From an early age we were told to pay attention and to concentrate on
whatever is put in front of us, be it a book, a monitor or a TV screen.
No doubt we became very skilled at accumulating information in this way,
but an unintended side effect may be to end up feeling as if we had only
a very small cramped space to live in, with our subjective sense of the
world becoming painfully closed and limited. At least in part, this is
why so many people reach a point where they feel they can’t take it any
more: ‘I haven’t got enough room to move!’ But this ‘room’ is a
fabrication, an imposition on awareness that we are \emph{doing}.

This perception of the personal space which we occupy is not a fixed
thing; we can work on dissolving those perceived limitations. Using our
imagination, we can expand the field of awareness. We can intentionally
generate a sense of broadening by suggesting to the mind, ‘\emph{Be
edgeless}’. As an experiment, bring attention to the temperature of the
air touching your skin. Then try feeling a few centimetres outside your
skin. Is it possible to sense the temperature of the air around your
body? Does it become warmer or cooler the further away you get? Or
experiment in the same way with sound. You can hear sounds immediately
next to you; now try listening to sounds a bit further away, then
further away again. Imagine listening, sensing, a very long way away.
What we are feeling for here is the ability to relax the sense of being
defined by a perception of rigid, limited space. Using our imagination,
we can create an image of a field of awareness expanding beyond the
immediate sensation of our body, outward, ever increasing, with the
suggestion, ‘\emph{Be edgeless}’. This is a field of awareness vast enough to
accommodate all of life. This is exercising ‘broadening’.

\section{Gently Listening}

If you can hear the high-frequency internal ringing of the sound of
silence, by gently listening to this sound you can discover a different
way of paying attention. When we send attention out through our eyes, we
easily narrow our field of awareness. We often equate paying attention
with excluding everything other than the object on which we are
focusing. This has its uses when intense concentration is what is called
for, but it is distinctly unhelpful when this way of paying attention
becomes our everyday mode of operating. It leads to an insensitive,
closed-off type of attention, not a skilful, sensitive attunement. If we
want to be able to see beyond the deluding stories that we have hitherto
believed, we need to be able to tune in sensitively to what life is
presenting to us. Being closed off and insensitive is the last thing we
need.

Turning attention to our ears and listening, away from our eyes and
looking, can relax the way in which we pay attention. Listening is a 360
degree application of attention. Listening is less ‘doing’ and more
‘allowing’; less ‘selecting’ and more ‘according with’. To~support
easing out of the picking and choosing mode, try suggesting to your
mind, ‘\emph{Gently listen}’. Intentionally listening in this way to the
sound of silence is cultivating a new disposition or attitude towards
experience. Instead of always controlling what appears in awareness and
trying to ‘get something’ out of experience, we simply open to what at
this moment is available, being willing to learn. This is exercising
‘gentle listening’.

\section{Simply Receiving}

When we have learnt to relax the way in which we pay attention and be
available to learn from everything that life offers us, this means we
have already loosened our grasp on compulsive tendencies to control. If
we keep checking to see whether we are still controlling or keep trying
not to control, that means we are still caught in controlling. It is
only when we have grown tired of deluded ego’s dishonest games that our
compulsive tendencies to want to be in charge fall away. We don’t drop
them by trying to drop them. Letting go happens when we see with insight
that clinging is fruitless. This is why the Buddha said, \emph{It is because
of not seeing two things that you stay stuck in saṃsāra: not seeing
suffering and not seeing the causes of suffering.} Trying to let go only
perpetuates the struggle. Rather, make the suggestion, ‘\emph{Simply
receive}’ to your mind. Trust that this receptivity has within it
the potential to see clearly, to understand, and that it is
understanding which brings about letting go. Don’t be afraid that
cultivating such sensitive receptivity will lead to a kind of passive
selfishness. When there is such a quality of awareness, any expression
of selfishness is more likely to be seen for what it is: a tired and
painful limitation that we are imposing on awareness.

\section{The Just-So Reality}

What we are being receptive to is the just-so reality of this moment.
If there is fear, receive fear into an expanded field of awareness and
allow fear to be ‘just so’. If there is anger, receive the anger and
allow it to be ‘just so’. If there is wanting, not-wanting, liking,
disliking, receive it all and accept that it is all just so. There are
causes for the conditions of this moment to appear as they do here and
now. Our task is to develop the quality of attention which means we can
receive this just-so reality honestly, nothing added and nothing taken
away. We are not programming ourselves to believe in the just-so
reality. As with the other prompts, the suggestion to recollect the
just-so reality supports honest, careful, receptivity of this moment.

\section{Finding Your Way}

Over the years I have witnessed many meditators trying to squeeze
themselves into forms which clearly don’t fit, so perhaps some will find
it helpful to know there is more than one way to climb a mountain.
Parents lovingly encourage their children to develop according to their
abilities. Alert to the individual needs of their children, parents give
them permission to experiment and to discover for themselves what works.
Wise yoga teachers warn their students against using force as they
become acquainted with the \emph{āsanas}. Hopefully, wise meditation teachers
will also tune into the individual abilities and needs of their
students, giving them the freedom to find out what works and encouraging
them to ask what is it that, for them, truly nourishes selfless confidence.

Thank you very much for your attention.
