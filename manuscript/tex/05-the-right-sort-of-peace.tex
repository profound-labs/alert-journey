\chapterNote{%
  ``\ldots\ instead of struggling against some activity which is taking place in
  awareness, try attending to the quality of awareness itself, the just-knowing
  mind.''}

\chapter{The Right Sort of Peace}

\chapterTitleVertical{T\\H\\E\par R\\I\\G\\H\\T\par S\\O\\R\\T\par O\\F\par P\\E\\A\\C\\E}

\tocChapterNote{Peacefulness, fixed positions, softness, gentleness, subtlety,
limitations, balance, stillness, ocean.}

There are some who approach Buddhism with the view that progress on this
journey is to be measured in terms of how peaceful they feel. Certain
circles take this further and speak of Buddhism as a form of quietism.
It is of course correct to say that the Buddha encouraged cultivating
tranquillity, but to suggest that the goal of Buddhist practice is to
hold onto peaceful states of mind is mistaken.

The Buddha didn’t want us to hold on to any particular state of being,
not even peaceful states. Of course, time spent in quietude can help us
do the demanding work of seeing beyond all the stories we tend to tell
ourselves about reality – the stories that indulging in pleasure will
make us happier, or that finding identity by clinging to the body will
lead to satisfaction. What the Buddha wanted, and what he consistently
taught, was that we need to know the reality of all states of mind, both
peaceful and not peaceful. If we want freedom, we must thoroughly
understand that holding fast to anything, any material thing, any story,
any view, any position, will inevitably lead to suffering. So Buddhism
is not about becoming peaceful.

It is not even about becoming enlightened. Some of you might have heard
that Ajahn Chah once commented, perhaps rather provocatively, ‘Don’t be
an arahant; don’t be a Bodhisattva; don’t be anything at all. If you are
anything at all you will suffer.’ This way of talking can be confusing
for those who hear the words, but not beyond the words to what is really
being said. The words that we use to help each other understand are
symbols; they are ways of pointing to the direction we need to go to
find what it is that we seek. Animals less intelligent than human beings
tend to look at a finger which is pointing to something, instead of
whatever the finger is indicating. We wouldn’t make such a mistake when
regarding activity in our outer world, but how about our inner world? We
must remember that we are not supposed to focus merely on the words
being spoken, but also on the understanding to which the words refer.
When Ajahn Chah says, ‘Don’t become anything at all’, he is pointing to
an understanding of the true nature of desire; he wants us to see
accurately that being caught up in becoming anything at all is a form of
clinging, and all clinging leads to suffering. It is by letting go of
any fixed position that we are freed.

And remember, it is all right if we don’t immediately understand what
the teacher is telling us; it is OK if we don’t immediately get the
message. Some of the Buddha’s disciples took a very long time before
they truly heard what he was saying. Even some of those living closest
to him never really got the message. What matters is that we persist in
our effort to listen to true teachings.

\section{Unobstructed Relationship}

In a previous talk I referred to the Buddha’s teaching known as The
Discourse on the Simile of the
Heartwood,\cite{mahasaropama-sutta} where we are encouraged to
keep letting go, whatever is happening in our lives. Don’t settle
anywhere, however comfortable or uncomfortable things might be, until
complete freedom is realized. So long as we live in a state of
unawareness, there is always the temptation to forget what it was that
inspired us to start out on this journey: that is, the possibility of
living in perfect, unobstructed relationship with all existence,
conditioned and unconditioned. So long as we still haven’t seen through
habits of clinging, we will be tempted to settle for something
superficial, including relatively peaceful states of mind. Being
attached to peacefulness is a bias, and as such obstructs our
relationship with reality. When the Buddha shared the fruits of his
awakening, he wanted us to know what he knew; that is, that once
awareness is free of any imposed limitation, when we have let go of all
biases, awareness can and will be able to accommodate whatever arises.
There is no level of intensity or mediocrity, no experience or
perception whatsoever, that can disturb the heart of one who is fully
awakened.

Our spiritual work, then, is to cultivate the whole body-mind capacity
that can accommodate everything and anything – peacefulness, aversion,
enthusiasm, despondency and any other state you might imagine – without
losing perspective, without causing suffering for ourselves or others.
We want to be able to look and listen so closely that we stop being
fooled. We want to see where the \emph{actual} causes of struggle lie. For
example, from a superficial perspective, when somebody says something
hurtful it looks as if they caused us to suffer. On closer inspection,
we find that what was said triggered the sense of pain, but it is
because of what we added to it that we suffered. It is because of our
resisting and indulging that we fail to see clearly and feel we have
problems. There are no real problems. If problems were ultimate in any
sense, freedom would not be possible. Regrettably, many people hold to
the belief that their problems and the problems of the world are
ultimately real; hence the terrible struggles. Indeed holding on tightly
to perceived problems is one of the ways in which many people find a
sense of identity. Awakened beings, on the other hand, know that
problems can appear to exist, but only because of the denial of reality;
all problems disappear when resisting and indulging ceases. The pain
doesn’t disappear, difficulties don’t disappear, but when suffering
disappears the natural pain of life is much easier to deal with.

\section{Nothing Lacking}

When we don’t look closely enough, when we fail to see beyond the
surface level, we can make the mistake of believing there is something
wrong with us, that we are somehow damaged or lacking. The idea of
attaining wisdom and compassion may be very appealing, but we think that
somehow we have to import them. The Buddha’s teaching does not promote
merely believing in wisdom and compassion, but neither does it tell us
that we are inherently damaged goods which need fixing. Nor, for that
matter, does the Buddha teach that somebody else has what we are looking
for and can bestow blessings upon us if we make sufficient
supplications.

With a trusting heart and a mind that is rightly directed, we can have
confidence that the wisdom and compassion we are looking for already
exist, there in the awareness which has been freed from the distortions
caused by clinging. It is a warp of awareness which makes us believe we
are somehow lacking, causing us to depend on what is not dependable. In
the \emph{Mahā-Maṅgala Sutta}\cite{mahamangala-sutta}
the Buddha instructed, \emph{pūjā ca pūjaneyyānaṃ},
which translates as ‘honour that which is worthy of
honour’, i.e. orient your life towards that which is truly reliable.
Wise reflection and listening to teachings from those who know truth can
give rise to the clear seeing which sees beyond false sources of
security. This clear seeing has the power to reveal the reality of
suffering. Then blaming anyone or anything simply doesn’t happen; there
is the understanding that blaming only happens when our hearts are held
closed in resistance to reality. Fear, anger, anxiety are all
expressions of this resistance. It is this closing of our hearts that
limits awareness, creating the impression that there is not enough room
for life. And it is not only painful feelings that cause us to struggle.
Pleasant feelings too can overwhelm us, resulting in feelings being
projected onto external objects, both people and things. We then feel we
have become dependent upon them, that we can’t live without them. If we
were already living in an open-hearted, trusting way, in an expanded
field of awareness, we wouldn’t have had the perception that ‘this is
all too much’, we wouldn’t feel overwhelmed. Projecting our heart’s
energy outwards wouldn’t occur. We would know that reality is never too
much; reality is always just so. The feeling we sometimes have that
things are all getting too much is the direct result of what we do that
imposes limitations on awareness.

\section{No Enemies}

Holding fast to any fixed position is a denial of reality. Clinging to
peaceful mind states and fighting off unpeaceful mind states only leads
to more confusion, causing us to feel we are surrounded by enemies and
life is always a struggle. By way of experiment, instead of struggling
against some activity which is taking place in awareness, try attending
to the quality of awareness itself, the just-knowing mind. Another way
of bringing about a shift in perspective is to observe the space round
an object instead of focusing on the object itself. If you can bring
about such a shift in perception you will find that confusion as a state
of mind is waiting to be received; like all forms of suffering,
confusion is something to be studied. We can learn a lot from confusion.
Confusion is not our enemy. When we learn to relate with life like this,
we won’t feel obliged to be always trying to become peaceful and
critical of that which is not peaceful.

Ajahn Chah once suggested that to see the maturity of a monk, you
shouldn’t observe him when he is sitting in meditation, but watch how he
conducts himself on a festival day. For most of the year life in a
forest monastery is quite quiet, with very little happening. But for
three or four days each year, festivals take place to mark events such
as the Buddha’s birthday or the beginning of the Rains Retreat. It is on
these occasions, Ajahn Chah was suggesting, when large crowds of
visitors come to the monastery and the senses are being bombarded with
sense objects – sights, sounds, smells, tastes – that you can tell
whether a monk has real spiritual ability. When external conditions are
agreeable, it is easier to feel peaceful and think we ‘have it
together’. When conditions challenge us, when we are driven to the edge
of our practice, to the growing tip, that is when we really learn. If we
hold to our preference for always feeling peaceful, we could miss the
opportunity to grow. Don’t be afraid of chaos; be afraid of how long it
takes to remember to be aware.

\section{Useful Skills}

It takes considerable skill to accord with conditions without resisting
and thereby causing suffering. But in mindfully receiving the
consequences of our \emph{not} according with conditions we can learn, as we
receive without any judgement the suffering we are creating. The
willingness to look more deeply, especially at those times when we feel
we are failing, serves the emerging understanding.

Take the everyday example of drying clothes in a spin-drier. If we
decide the drier has been running long enough and turn the machine off,
that doesn’t mean the barrel will immediately stop spinning. The
movement of the barrel has momentum to it. If you put your hand in and
try to force it to stop, you could get hurt. Similar common sense
applies when we light a fire in a wood-burning stove: even though it
might look black and harmless on the outside, we wouldn’t be tempted to
put our hand on the stove. Our understanding protects us from getting
too close and burning ourselves. Children who haven’t yet learnt the
need to take care depend on their parents to protect them, hence the
fireguards. Once children have learnt the lesson about the risks of
getting burnt, they don’t need fireguards.

We need to protect our hearts from what we do with life that turns it
into suffering. We need to understand that it is not life which causes
suffering, it is the way we relate to it. When there is right
understanding we are careful. When we are not careful, when we are
heedless, we close our hearts, we limit awareness with our habits of
clinging, and then feel that we are somehow inherently lacking. But we
are not lacking; we only create the impression that we are. And then we
believe that we don’t have enough of everything: not enough patience to
endure the unendurable, not enough kindness in the face of enmity, not
enough perspective to accommodate ambiguity; that we don’t have enough
room for life. But the good news is that if we can create such an
impression, of course we can also cease to create it.

Everybody on this journey forgets from time to time and reverts to
habits of clinging. Hopefully the effort we make means we are learning
to remember more quickly. Softening our approach to life, being more
gentle, more careful, not assuming too much about the way things appear
to be on the surface, means that sensitivity matures, nurturing insight.
This softness, this sensitivity, is not a form of weakness. When we
genuinely admit to how life affects us, without indulging or denying, we
grow stronger. The right kind of gentleness leads to a flexible sort of
strength, not to increased rigidity. In turn, it supports clarity. As
strength and clarity develop, we grow more confident in receiving
everything, accommodating everything and learning from everything. This
is a very different approach to spiritual practice from one that judges
peacefulness as a sign of success and the absence of peace as a sign of
failure.

\section{Stillness in the Depths}

A state of relative peace of mind is like the ocean without waves or a
lake without ripples. When the surface of the lake is still you can see
a beautiful reflection, one not there when the wind is blowing and the
surface is disturbed. The beauty of that reflection is like the pleasure
of a mind without too many disturbing thoughts or mental impressions.
However, we don’t expect the lake to always be still, or the ocean to
always be without waves. And it is not sensible to expect our minds to
always be peaceful. If we have the facility to access such relative
tranquillity, we will know the state of joy and ease that can be found
there. But we must also know that these states of mind, like the
reflection on a lake, come and go and we are careful to not allow them
to lead to attachment.

There is another type of peacefulness with which we would be wise to
acquaint ourselves. As with the stillness which is always there at the
bottom of the ocean and remains undisturbed by the activity above, we
can trust that deep within us, there is a dimension of peacefulness
which is always there. As practice progresses, an initial quality of
trust can evolve into a confidence born of insight. The stillness at the
bottom of the ocean is unperturbed even when massive breakers are
crashing about on the surface. We can afford to trust that
within is a deep stillness which is `just there',
beneath all the activity. This is a peacefulness
that doesn’t require propping up or sustaining.

If we have some sense of the stillness which is always there, we are
less likely to mistake surface turmoil for being anything more than the
changing nature of things. When we appreciate the relativity of turmoil
there is less chance of infatuation with the drama of the world; we are
more interested in seeing beyond the way things appear to be. There is
no end to the waves on an ocean; they are a natural expression of the
ocean. It wouldn’t be wise to want to stop oceans from having waves. And
it is not wise to demand that our minds always be peaceful. When we shed
that attitude, we feel more able to accept the forever changing nature
of things. It is easier to surrender our resistance to what we don’t
like and avoid getting lost in what we do like. We stop struggling to
change the nature of the world, and work instead on our relationship
with the world. When we lose ourselves in the surface turmoil, we tend
to incline towards distraction or despair and start complaining that it
shouldn’t be this way. When we understand the nature of the world
accurately we can accord with it, and have a better chance of generating
real benefit.

Pointing out the fruitlessness of complaining is not to say we shouldn’t
do anything. To point to the futility of trying to change the nature of
the world is not to advocate apathy. Quite the opposite! Developing the
agility of attention which means we have access to stillness when it is
needed and the capacity to accord with activity when it is called for,
is being responsible. We are positioning ourselves with optimal
perspective, so as to see where and when we become stuck, creating the
unnecessary impression of having problems. It is in letting go of our
attachments to ‘me’ and ‘my way’ that we can make a real difference and
allow natural selfless goodness to shine.

\section{Contributing Well-Being}

If we can’t unplug from always pursuing preferences, we limit what we
can contribute. Clinging to being peaceful and resisting that which
disturbs us leads to stress. One of the best ways to increase well-being
for ourselves and others is to cultivate mindful agility. Viewing the
world from contrasting perspectives can give rise to insight. Getting to
know ourselves, both in the midst of peace and tranquillity, and when we
are surrounded by irritating and annoying conditions, help us grow. Just
as the developing intelligence of a child is stimulated by experiencing
contrasting colours, textures and environments, so the accuracy of our
view of the world is enhanced by experiencing contrasting perspectives.
The richness of a painting, the depth of a photograph, the impact of a
piece of music, all depend on contrast. So long as we are attached to
being peaceful and reject what is not peaceful, we bolster the divisions
in our world. We risk making the perceptions of separateness – ‘us’ and
‘them’, ‘me’ and ‘mine’ – even more rigid. That certainly doesn’t help.
If we have trained our minds to sustain clarity and kindness in the
context of both calm and chaos, we are more likely to see beyond our
conditioned preferences to that which is truly beneficial. This agility
of attention helps us discern new ways of handling the chaos, of not
being intimidated by how troubled our inner and outer worlds sometimes
appear to be.

Thank you very much for your attention.
